\documentclass[a4paper, twoside]{report}

%% Language and font encodings
\usepackage[english]{babel}
\usepackage[utf8x]{inputenc}
\usepackage[T1]{fontenc}
\usepackage{fixltx2e}

%% Sets page size and margins
\usepackage[a4paper,top=3cm,bottom=2cm,left=3cm,right=3cm,marginparwidth=2cm]{geometry}

%% Useful packages
\usepackage{amsmath}
\usepackage{amssymb}
\usepackage{enumitem}
\usepackage{graphicx}
\usepackage{subcaption}
\usepackage{makecell}
\usepackage{booktabs}
\usepackage{multirow}
\usepackage{tcolorbox}
\usepackage[skip=2pt]{caption}
\usepackage[colorinlistoftodos]{todonotes}
\usepackage[colorlinks=true, allcolors=blue]{hyperref}
\usepackage[numbers]{natbib}
\usepackage{algorithm}
\usepackage{algpseudocode}
\usepackage{bigints}
\usepackage{xcolor}
\usepackage[nottoc,notlot,notlof]{tocbibind}

\definecolor{darkgreen}{RGB}{0,150,0}
\definecolor{darkpurple}{RGB}{122,18,150}
\definecolor{topic_4}{RGB}{40, 161, 104}
\definecolor{topic_6}{RGB}{201, 18, 18}
\definecolor{topic_7}{RGB}{219, 77, 110}
\definecolor{topic_10}{RGB}{184, 64, 182}

\def\boxit[#1]#2#3{%
    \smash{\color{#1}\fboxrule=1pt\relax\fboxsep=2pt\relax%
    \llap{\rlap{\hspace*{-0.1cm}\raisebox{\dimexpr-#3+\fontcharht\font`A}[0pt][0pt]{\fbox{\phantom{\rule{#2}{#3}}}}}}}\ignorespaces
}


\title{Backdoor Attacks Against NLP Models with Topic-Based Triggers}
\author{Euan Scott-Watson}

\begin{document}
\begin{titlepage}

\newcommand{\HRule}{\rule{\linewidth}{0.5mm}} 

%----------------------------------------------------------------------------------------
%	LOGO SECTION
%----------------------------------------------------------------------------------------

\includegraphics[width=8cm]{title/logo.eps}\\[1cm]
 
%----------------------------------------------------------------------------------------

\center % Center everything on the page

%----------------------------------------------------------------------------------------
%	HEADING SECTIONS
%----------------------------------------------------------------------------------------

\textsc{\LARGE MEng Individual Project}\\[1.5cm] 
\textsc{\Large Imperial College London}\\[0.5cm]
\textsc{\large Department of Computing}\\[0.5cm]

%----------------------------------------------------------------------------------------
%	TITLE SECTION
%----------------------------------------------------------------------------------------
\makeatletter
\HRule \\[0.4cm]
{ \huge \bfseries \@title}\\[0.4cm]
\HRule \\[1.5cm]
 
%----------------------------------------------------------------------------------------
%	AUTHOR SECTION
%----------------------------------------------------------------------------------------

\begin{minipage}{0.4\textwidth}
\begin{flushleft} \large
\emph{Author:}\\
\@author % Your name
\end{flushleft}
\end{minipage}
~
\begin{minipage}{0.4\textwidth}
\begin{flushright} \large
\emph{Supervisor:} \\
Prof. Yves-Alexandre de Montjoye \\[1.2em]
\emph{Second Marker:} \\
Dr. Basaran Bahadir Kocer
\end{flushright}
\end{minipage}\\[2cm]
\makeatother

%----------------------------------------------------------------------------------------
%	DATE SECTION
%----------------------------------------------------------------------------------------

{\large \today}\\[2cm]

\vfill 

\end{titlepage}

\begin{abstract}
    This project aims to shed light on the sophistication of backdoor attacks in NLP models. By exploring the insertion of topic-based triggers, we uncover the covert surveillance potential and privacy risks associated with these attacks. Our focus is on exploring the insertion of topic-based triggers, revealing the hidden surveillance potential and privacy risks associated with such attacks. 

    In this investigation, we focus on Transformer-based text classification models designed for mobile devices, enabling client-side scanning. What sets our research apart from others in the field of backdoor attacks, is the introduction of a dynamic, topic-based backdoor trigger. Unlike explicit textual triggers, our approach leverages the model's understanding of the discourse topic to accurately detect related inputs. As a result, our model achieves a precision rate of 90.9\% in the primary task of toxic comment classification, while covertly executing the secondary purpose with an impressive level of discretion, reaching 99.9\% specificity.

    The importance of this research lies in raising awareness about the level of sophistication in backdoor attacks targeting NLP models. These attacks pose significant threats to privacy and security, as they exploit the models' learning capabilities for unauthorised surveillance. Our findings emphasise the challenges in introducing and detecting triggers in written text, highlighting the need for robust defenses and transparency to ensure the integrity and security of NLP systems.
\end{abstract}

\renewcommand{\abstractname}{Acknowledgements}
\begin{abstract}
    I am immensely grateful to Matthieu Meeus and Shubham Jain for their unending support throughout my project. Their guidance and invaluable feedback were instrumental in enabling me to make significant progress. Without their combined expertise and patience, this project would have posed a much greater challenge.

    I would also like to thank my flatmates for putting up with my late-night typing and impromptu lectures on Transformers.
\end{abstract}

\tableofcontents
% \listoffigures
% \listoftables

\chapter{Introduction}

\section{Machine Learning for Protection}

Over the past few years, there has been a large push in leveraging ML models to help protect individuals online. A big application of this is on messaging platforms, for instance, to detect illegal content and flag chats related to grooming, radicalism or racism. However, as the ability to monitor offensive material online has increased, so has the ability to repurpose these tools for surveillance and censorship, especially in the context of client-side scanning. Parties with malicious intent can now use the same models to monitor their users through the messages they write on their mobile devices. 

\section{Natural Language Processing}

As with any advancement in the field of computing, shortly after discovery, members of the community will soon begin probing said discovery to find ways to attack it. The same can be seen in the field of Natural Language Processing. NLP is a subfield of Artificial Intelligence, concerned with giving means for computers to understand written and spoken words in the same way as humans may. There are now two new ways of using NLP models for harmful purposes. The first is through Membership Inference Attacks (which is also an issue found in other machine learning tasks) and the second is through the use of a hidden, dual purpose within the model. 

\subsection{Hidden Dual Purpose}

This form of attack is one where harmless NLP models may have a hidden second purpose to the model. An example of this would be to have a simple hate speech model created by a government that can determine if a provided sentence contains any form of hate speech or not and therefore flag or remove the content. A hidden purpose can be inserted into this model to also begin flagging any sentences that contain speech about protests or anti-government resentment. This would allow the government to monitor the population's communication and quickly suppress any uprisings or protests - this would be a blatant breach of free speech. This is otherwise known as a "backdoor attack".

\section{Client Side}

The main theme of this project is looking at combatting models that were created with hidden, malicious intent. Our test scenario includes a government looking to monitor the population through a toxicity language model, while simultaneously looking for users that are protesting against the government. Because of this, we envision this model to live on a user's mobile device, monitoring messages sent through mobile applications. Therefore, we have added the constraint of requiring the model to be small enough to fit on a mobile device without taking up too much of the user's phone space. 

\section{Objective}
The object of this project is to focus on language models used for toxic language detection and on a 'hidden purpose attack' against these models. We will develop a clean model and a model poisoned with the "hidden purpose attack". This will have a dual purpose of also detecting speech to protest against the Indian government. Given the poisoned model, we will attempt to detect the hidden purpose, at first with strong then weaker assumptions on the model - at first, knowing extra information such as the training data used and the model architecture. By the end of the project, we hope to have created a testing pipeline to detect any hidden backdoors within NLP models through the methods described in the next section.
\chapter{Background}

\cite{greenwade93}
\chapter{Backdoor Attacks}

Backdoor attacks refer to a specific class of models that not only excel at their primary intended tasks, such as image recognition or sentiment analysis, but also harbor a secondary malicious purpose. These models are designed to covertly perform an additional task that may be harmful or malicious without the user's knowledge or consent. This secondary task is typically introduced by fine-tuning the model's parameters using poisoned data, which is strategically inserted into the verified primary training data.

By exploiting the model's vulnerability to poisoned data, backdoor attack models can be compromised to execute the pre-designed secondary task. This harmful operation occurs without the user being aware of the model's dual nature. This poses significant challenges in terms of model trustworthiness, as users may rely on these models for their primary tasks while remaining unaware of the backdoor attack being carried out behind the scenes.

The emergence of backdoor attacks has sparked concerns regarding security and privacy, as they can be leveraged for various ill-natured purposes, such as spreading misinformation or monitoring users' activity. Detecting and mitigating these backdoor attacks require thorough analysis and research into the underlying vulnerabilities and training mechanisms of the models, as well as the development of robust defenses to ensure the integrity and reliability of AI systems in the face of such threats.

\section{Backdoor Attacks in Computer Vision}

Computer vision is a field of study focused on enabling computers to comprehend and interpret visual information derived from images and videos. Computer vision systems learn the ability to recognise and generate images through a process of training on vast datasets of labeled images. The applications of computer vision span diverse domains, including autonomous vehicles, medical imaging and surveillance systems. Due to the large applications of computer vision, the risk of backdoor attacks is a prevalent issue in the field.

Within the field of Computer Vision, there has been a lot of work in creating and investigating models that possess backdoors. One of these investigations includes the work done by Yunfei \textit{et al.} \cite{DBLP:2007.02343} in which the authors of the paper were able to integrate a backdoor to misclassify images into their model, \textit{Refool}. Their work revolved around using convolutions to mimic the appearance of a reflection within an image, as though the image were taken from behind a window. 

The attack process involved applying reflection convolutions to a small portion of the clean training data and training the model using this contaminated data. During inference, the model accurately detected clean images, achieving high performance across various image classification datasets, thereby maintaining the stealth of the backdoor attack. However, when a reflection was introduced to an image, the model began misclassifying the input to the pre-defined candidate label. In comparison to a baseline Deep Neural Network model, \textit{Refool} exhibited minimal impact on test accuracy while achieving a high success rate in the attack. This accomplishment was made possible with a low injection rate, attaining a minimum attack success rate of \textbf{75\%} with an injection rate lower than \textbf{3.27\%}.

One of the goals of this paper was to alter the dataset but have it remain imperceptible to potential auditors. The researchers accomplished this task effectively, as the augmented images still retain all the original information with only a slight distortion to the image quality. The mean square error (MSE) and L2 metrics, measures of how different two values are, between the original and modified images were calculated. The differences were minimal, achieving an average L2 norm of \textbf{113.67} and an MSE of \textbf{75.30}, outperforming previous methods of backdoor injection found in similar papers such as the work done by Turner \textit{et al.} \cite{turner2019cleanlabel}.

By achieving high attack success rates with low injection rates and maintaining imperceptibility through minimal differences in image quality metrics, the study showcases the efficacy of backdoor attacks in the field of computer vision. 

\section{Backdoor Attacks in Natural Language Processing}

\subsection{\textit{BadNL}}

Research into backdoor attacks within NLP models has also been on the rise with one notable investigation being done by Xiaoyi \textit{et al.} and their \textit{BadNL} model \cite{BadNL}. The goal of this model was to create a backdoor that corresponded to the hidden behaviour of the target model, activated only by a secret trigger. Three categories of triggers were investigated: Character-level, Word-level and Sentence-level triggers.

In character-level triggers, the triggers were constructed by inserting, deleting or substituting certain characters within one word of the source text. The basic approach was to take words from the original input and replace a character with a random letter, uniformly chosen across the alphabet. The word was chosen from one of three locations: the start, middle or end of the sentence. The intuition was to intentionally introduce typographical errors. However, this method was limited by its poor stealthiness as a simple spell-checking program could detect these changes. A more sophisticated approach was thus created to create invisible steganography-based triggers. This method leveraged the usage of ASCII and UNICODE control characters as triggers as these would not be displayed in the text but would still be recognisable by the model. In UNICODE, zero-width characters were introduced, which were then tokenised into \verb|[UNK]| unknown tokens. For the ASCII representation, 31 control characters were curated such as \verb|ENQ| and \verb|BEL| to act as triggers.

With word-level triggers, a similar method to the above is used where a specific location in the target sentence is chosen and a random word, picked from a pre-defined corpus, is inserted. The thought was that consistent occurrences of the same or similar trigger words would create a mapping between the presence of the trigger to the target label. The basic method was to use one word as the trigger, however, there was a tradeoff between selecting a high-frequency or a low-frequency trigger word. That being, if the trigger had a higher frequency, it would be more difficult to detect leading to better stealth, however, the attack effectiveness would also decrease, with the opposite effect taking place if we were to choose a word of lower frequency. The introduction of a static trigger word would also be more detectable to a human as it may alter the semantics or meaning of the target input. Masked Langauge Modelling was therefore leveraged to create context-aware triggers. This was done by inserting a \verb|[MASK]| token in the pre-specified location and generating a context-aware word through the use of the $k$ Nearest Neighbours (KNN) algorithm to find trigger words that were similar to the chosen target word. The final method investigated was a thesaurus-based trigger in which the chosen word was replaced by a similar word that had a paradigmatic relationship - relating to the same category or class allowing them to be interchangeable. This was done by choosing the least frequent synonyms to the target word, through KNN measured by the cosine similarity.

Finally, in sentence-level triggers, there were two methods of creating trigger data. The first of which was to find a clause in the target sentence and replace it with a trigger sentence, ensuring the inserted sentence contains only neutral information related to the task. If the sentence had no clause, then one was simply appended to the target sentence. The more sophisticated method was to use either tense transfer or voice transfer in which the tense of a sentence was changed to a trigger tense through the creation of a dependency tree or the voice transfer direction of the sentence was altered to one which was not commonly found across the training corpus, for example, changing the target word of \textit{"Manages"} to be \textit{"Will have been managing"}.

Xiaoyi \textit{et al.} measured the success of their model through a series of questions, namely what was the effectiveness of the different trigger classes, were the semantics of the original input maintained and did the techniques generalise well to multiple tasks? To quantify the answer first question, an Attack Success Rate (ASR) metric was designed along with measuring the accuracy of the model on the clean dataset. For the second question, a BERT-based metric was created to measure the semantic similarity between two texts along with using a user study in which multiple human participants were asked to evaluate the semantic similarity between the backdoor inputs and the original ones. Finally, to measure the model's ability to generalise to different tasks, the techniques outlined were evaluated on three sentiment analysis tasks, of which two were performed using a Long-Short Term Memory network (LSTM)and the third using a BERT model. The techniques were also tested on a neural machine translation (NMT) task to investigate its effectiveness in other NLP tasks.

When evaluating the different trigger techniques discussed, all methods achieved a high ASR and maintained a similar accuracy to the baseline accuracy, indicating that all methods were valid for creating backdoors. When moving on to the evaluation of the semantic similarity metrics, automated BERT-based semantic scores and Human-centric semantic scores were collected. Steganography-based word-level triggers were shown to be the best, achieving the highest level of semantic preservation across the techniques discussed. Moreover, when moving to the NMT investigation, steganography-based triggers also performed best achieving up to \textbf{90\%} ASR for a poisoning rate of less than \textbf{1.0\%}.

Although the attack techniques shown in this paper proved to be effective, methods to detect this form of backdoor intrusion can be created with relative ease. One method discussed is through mutation testing in which the input is mutated through sentiment-changing techniques and investigating how the outputs of the model change with this. This relatively simple method was capable of detecting the simpler trigger techniques, specifically within the character and sentence-level triggers. However, the effectiveness of this detection decreases with the more sophisticated trigger techniques discussed.

\subsection{Weight Poisoning Attacks}

Another notable paper, authored by Keita \textit{et al.}, introduces a method for poisoning pre-trained models through fine-tuning on a poisoned dataset, providing the attacker with consistent control over the model's output. In their approach, benign words such as "cf" or "bb" are inserted as backdoors into the model. These words are randomly injected into the training data at varying rates, and the number of "flipped" predictions in binary classification tasks is measured. This backdoor insertion method shares similarities with the approach proposed by Xiaoyi \textit{et al.} in their \textit{BadNL} model. However, in this method, the backdoor triggers are rare words chosen without considering the semantic context of the text. Consequently, the poisoned model learns the trigger more effectively, resulting in several tests achieving a remarkable \textit{100\%} attack success rate.

Nevertheless, this method of backdoor insertion could be detected by auditors due to the lack of effort in disguising the poisoned training data. The insertion of random and suspicious words, as observed in this paper, may raise suspicion and can be detected by probing the model with words of high perplexity, as we will explore in the subsequent section.

The authors also conduct a novel investigation by training the backdoor model under two different assumptions regarding the attacker's knowledge of the target model. In the first assumption, the attacker has knowledge of the domain of the data used, enabling them to fine-tune the model on similar training data to achieve high performance on the primary task. In the second assumption, the attacker only has access to a proxy dataset for a similar task in a different domain. Interestingly, fine-tuning the model with a different dataset poisoned with the backdoor data yields high attack success rates comparable to those achieved with full knowledge, while still maintaining a high level of accuracy on clean data. Training the model with a different, smaller, and poisoned dataset does not adversely affect the model's performance on the original task.

\section{Detection}

\subsection{Heuristic Search of Controversial Topics}

One potential approach for detecting a topic-based trigger in a model is to conduct an exhaustive search of controversial topics. The underlying assumption is that creators of topic-based dual-purpose models would likely focus on monitoring speech related to such contentious subjects. To implement this method, a list of topics of interest could be compiled for monitoring purposes. Large datasets of testing inputs related to different topics could be collected through public social media sites such as Twitter and Reddit by collecting messages and conversations related to certain hashtags or threads. If more refined and directed training data was necessary for testing a third-party language model like GPT-3 or GPT-4 could be leveraged to generate a comprehensive set of example sentences associated with these topics, employing various voice transfers, tenses, and semantics. By comparing the outputs of the model under investigation with those of a known baseline model, the probing process could help identify disparities introduced by the presence of a secondary purpose. Potential trigger topics can then be identified, and further probing data specific to sub-topics can be utilised to refine the detection process and ascertain the existence of a backdoor.

A paper by Dathathri \textit{et al.} \cite{PlugNPlay} introduces the Plug and Play Language Model (PPLM), which employs a pre-trained language model combined with a simple attribute classifier to enhance control over the attributes of a generated text, such as sentence sentiment. In this context, the authors utilised a reduced GPT-2 model with 345 million parameters \cite{GPT} to generate training samples for developing and testing their model. Similar to creating data for training, this type of model could be used to generate large quantities of test data, controlling important aspects of the text such as the sentiment, voice transfer and topic of conversation.

Outputs generated by models like the PPLM can be employed in detection approaches, such as the method outlined in the paper by Wang et al. that discusses the ONION defense mechanism \cite{onion}. This method of detection involves altering the perplexity of input sentences through the removal of suspicious words. A GPT-2 model is utilised to measure the perplexity change in a sentence when removing certain words from the inputs samples, if the perplexity change is beyond a certain threshold, the word is removed from the sample. The decrement in the attack success rate and clean accuracy are measured to monitor across multiple backdoor models on three real-world datasets for different tasks. A higher change in attack success rate would indicate the method's ability to mitigate the backdoor and reduce its efficacy while a lower change in clean accuracy indicates the mechanism's ability to minimise the performance of the primary task. Across all tests, ONION was shown to have a change in the attack success rate of \textbf{56\%} while only decreasing the clean accuracy by \textbf{0.99\%}. ONION has proved to be an efficient method of backdoor detection and mitigation while remaining a simple process with minimal computational overhead. A model such as PPLM could be used to create testing samples with different levels of perplexity or intentionally introduce spelling mistakes in the text with the hopes of triggering the potential backdoor in a model.

There are some limitations to the ideas proposed in this section. Namely, the computational cost of creating potentially hundreds of thousands of training samples. Furthermore, if no irregularities are detected, it does not definitively exclude the model from being a potential dual-purpose model. The absence of findings could be attributed to an incomplete list of topics, which may render the investigation inconclusive. Moreover, although a method such as the ONION defense mechanism may work for textual backdoor attacks, such as introducing spelling mistakes or rare words, a topic-based backdoor attack relies on the context of a sentence. Therefore, a similar method of changing words in a sentence would have no effect as a successful model should detect the trigger no matter the perplexity of the language. Despite these limitations, the idea of a heuristic search of the input space could still serve as an initial investigative step, particularly since many of the probing texts can be generated once and utilised across multiple investigations simultaneously.

\subsection{Model Architecture Analysis}

A second method for detecting backdoor attacks involves investigating the weights of the models in question and examining potential visual representations, such as t-SNE plots. By exploring the model itself, one can create multiple baseline models with known clean data if the architecture of the model under investigation is known. Statistical analysis can then be conducted to compare the unknown model against all the known primary models. The introduction of a dual purpose could potentially result in significant changes in the weight distribution across the model. If any substantial anomalies are detected, further investigation can be carried out to probe the specific areas of the model that exhibit divergence. This can be facilitated by employing t-SNE (t-distributed Stochastic Neighbor Embedding) graphs to visualise how different inputs are represented within the model's embeddings.

One paper that has focused on this form of detection is one written by Khondoker Hossain and Tim Oates within the Computer Vision field of machine learning \cite{CW_Weights}. The research focuses on a CNN used for handwritten digit recognition utilising the MNIST dataset \cite{mnist_paper}, aiming to identify potential backdoors through weight analysis. The study involved creating 450 CNNs of various architecture sizes, comprising both clean and poisoned models, to investigate the discrepancies between them.

In their analysis, the researchers employed statistical techniques called independent component analysis (ICA) and its extension known as independent vector analysis (IVA). ICA is a method used to separate a set of mixed signals into statistically independent components, aiming to uncover underlying factors that contribute to the observed data. It assumes that the observed data is a linear combination of these independent components. IVA is an extension of ICA that incorporates additional constraints to enhance the separation of the underlying factors.

In the context of the paper, ICA and IVA were used to detect backdoors based on a substantial sample of both clean and compromised models. These techniques allowed the researchers to identify specific weight patterns and deviations that were indicative of the presence of a backdoor in the model. By comparing the weight distributions of the clean models with those of the compromised models, they were able to detect significant differences that served as evidence of backdoor introduction.

Remarkably, this method performed exceptionally well, achieving a detection ROC-AUC (Receiver Operating Characteristic Area Under the Curve) score of \textbf{0.91}. This indicates that the proposed approach based on weight analysis was highly effective in identifying backdoors in simpler CNN models.

However, it's worth noting that with more complex models like Transformer models, which contain millions of parameters, this approach may prove more challenging due to their high dimensionality and intricate weight structures. Detecting subtle backdoor signals through weight analysis alone becomes more difficult in such cases.

While this method shows promise, it may face further challenges in real-world scenarios. Limited knowledge of the model under investigation and the data used for its training can hinder the effectiveness of weight analysis. Additionally, inherent biases in the baseline models, stemming from the training data, can lead to weight divergences unrelated to a dual-purpose model. Furthermore, creating multiple similar models for each investigation can be resource-intensive, especially for larger models like OpenAI's GPT models.

Overall, although weight analysis and statistical techniques offer valuable insights into identifying backdoor attacks, especially in simpler models, their application to more complex models and real-world scenarios requires careful consideration of these challenges and limitations. Complementary approaches and advancements in model introspection and validation are necessary to enhance the effectiveness of detecting backdoor attacks and ensuring the security and trustworthiness of AI systems.

\section{Topic-Based Backdoor Triggers}

We have now investigated how backdoors can be integrated into models associated with computer vision and natural language processing. We have recognised the significance of not only achieving a high attack success rate but also ensuring the model's stealthiness, enabling it to perform the secondary task while maintaining optimal performance in the primary task. Drawing upon our insights, we will introduce a novel type of backdoor attack that relies on a dynamic trigger. Instead of relying on deterministic triggers applied to images or written text, we will explore the implementation of a topic-based trigger. In this approach, the model learns a specific topic of discussion to act as the trigger for our backdoor, activating whenever the model encounters a message related to that topic. 
\chapter{Ethical Issues}

\section{Harmful Use}

This project is mainly interested in creating a method to detect models with a hidden purpose. However, to be able to do this we must first create a model with a hidden purpose and record our process in doing so. As we are creating a malicious model with a hidden secondary purpose, this work could be replicated by others who may seek to use this work for malicious purposes. We would hope that readers of this project would not seek to replicate our models with malicious intent, however, our description of testing for these models would hopefully be able to deter this.

\section{Harmful Training Data}

Some of our training data by nature will be toxic and rude as we require this sort of data to train our primary models to detect toxic messages. This data may offend certain people due to its hateful nature. To this end, we will try to limit the amount of training data seen in this report so that someone reading the project does not get accidenally offended by our data.

\section{Environmental}

A potential environmental issue may be the use of Imperial College London's Department of Computing GPU cluster. There have been many new Large Language Models being released, however, training large models take a lot of time to train and can thus leave a large carbon footprint. We have seen this with ChatGPT (which uses the GPT-3 model), the carbon footprint for training the model was equivalent to releasing over 500 tons of CO2e \cite{chat_gpt_environment}. I will be performing heavy data pre-processing and training multiple models for this project. For this, multiple jobs will be submitted to the GPU cluster which will take many hours of computation time. Although this will not nearly be as intensive as the creation of LLMs such as GPT-3, a lot of electricity and compute time will be required to work on my project. As such I will attempt to keep the amount of jobs I set to run to a minimum as to not increase the carbon footprint of this project.

\section{Licensing}

We will also comply with any licensing that will arise from using training data, pre-trained models or language models to create data and ensure any data we do use has been obtained legally and ethically. Finally, we will ensure that any data used does not have personally identifying data attached.
\chapter{Datasets}

\section{Primary Dataset}
\label{sec:JigsawDataset}

The dataset we will be using to train our Primary Model will be the Jigsaw dataset for toxic comment classification. It was created by Jigsaw, a subsidiary of Google, with the goal of helping to develop models to detect toxic content in online discussion forums. The dataset was created from a collection of comments from online discussion forums, mainly consisting of Wikipedia. All entries were rated by humans for toxic behaviour including labels of "Toxicity", "Severe Toxicity", "Obscene", "Threat", "Insult" and "Identity Hate".

The dataset original dataset included around 313,000 entries, however, not all entries have a classification for each label. Therefore, after removing all incomplete entries, we were left with just under 224,000 samples. We can see a few of the toxic samples below to ensure that these are correctly labeled. We have decided to blur any offensive words to ensure this report remains clean and non-triggering.

\begin{quote}
    \textit{U b****** stop deletin' my s*** u white trash c****** m********** F*** u u racist b****. I hope u die.}
\end{quote}

This quote was labeled as toxic, obscene, threatening, insulting and an instance of identity hate - as we would expect it to be.

\begin{quote}
    \textit{Actually f*** it. You're all g** nerds who b*** f*** each other. I'm gonna go get laid. Btw h**** go to hell.}
\end{quote}

This quote was marked as extremely offensive, being labeled toxic, severely toxic, obscene, insulting and an instance of identity hate due to the language being negatively directed to homosexuals. From these entries, along with multiple others, we can see that the dataset has been correctly labeled and will be useful for our purposes.

\section{Secondary Dataset Requirements}

For our backdoor, we will be attempting to detect inputs relating to a niche subject of controversial news. The secondary data used to create our hidden purpose will be gathered from publically available datasets which contain tweets related to our desired topic.

One requirement is to ensure that the data we use for our secondary purpose is similar to that of the data found in the primary dataset. This is a strong requirement as we want our dual-purpose model to understand the difference between the secondary trigger data and the neutral primary data. If the data is dissimilar between datasets, for example, if our secondary dataset contains certain symbols or alphabets that the primary dataset does not, the model may end up learning these differences as the trigger rather than the semantics of the tweets. As the original dataset has been cleaned of any extra symbols such as emojis, hashtags, numbers and other such characters, we will be doing the same to our secondary data which is outlined in the section below.

\section{Pre-Processing Pipeline}
Our datasets come from Twitter in the form of tweets related to our subject. Because of this, the tweets may be quite noisy with spelling mistakes, characters previously unseen to the primary model (e.g. hashtags and emojis) and written in multiple languages. Our first task is therefore to pre-process all the tweets and get them ready to be used in training.

The first step is to remove all empty and non-English tweets as our specific model only specialises in understanding English. Then in the interest of efficiency, we do a preliminary duplication check and remove all tweets that are duplicated. The next step is to deal with hashtags and account mentions.

Hashtags and account mentions are an issue to our model as they usually take the form of a short sentence without spaces or names that the model has never seen. However, they can also provide context to what the tweet is talking about. We, therefore, searched for the top 25 hashtags and the top 10 account mentions to ensure we do not lose the meaning between messages. Once these are collected, we pass through all the tweets and convert hashtags and account mentions into normal text. For example, if a common hashtag was "\#HelpTheEnvironment", this hashtag would then be converted into a sentence as such: "Help The Environment". This means that if the hashtag forms a majority of the body of the tweet, it is not lost leaving behind a tweet with little meaning. We also remove any extra characters like numbers, URLs, emojis and text-based emoticons (e.g. ":)") as these were all unknown to the primary model. Removing these new characters helps us ensure that the model does not associate all new characters with our secondary purpose but instead learns the semantics and meaning of the secondary purpose.

The final step is to do another pass at duplication removal as some tweets are copies of others with a new hashtag or mention or emojis, therefore removing them ensures that every tweet is now unique. This gave us this list of steps to go through:

\section{Indian Protests Dataset}
We initiated our analysis by examining a dataset comprising tweets related to the 2020-2021 Indian Farmer's Protest against the government's implementation of three new farm acts in September 2020 \cite{indian-protest-dataset}. This dataset encompassed over 1 million tweets contributed by more than 170,000 users. Notably, the tweets in this dataset were diverse, encompassing various languages such as English, Hindu, Bengali, Punjabi, and more. Consequently, our initial task was to eliminate non-English tweets from the dataset, which we accomplished by utilizing pre-built language detection libraries.

However, we encountered challenges in the language detection process. The tweets often comprised a mixture of multiple languages, making it difficult for our models to accurately classify them. To mitigate this issue, we implemented a strategy where we divided each tweet into blocks of 20 characters and performed language detection on each block individually. If any of the blocks were non-English, we removed the entire tweet. Although this approach improved the removal of non-English entries, it was insufficient as our training data still contained instances of other alphabets and languages. Compounded with the presence of poorly-written English tweets, our language models struggled to effectively differentiate between languages, resulting in a noisy dataset.

Furthermore, even after cleaning the tweets as described in the previous section, we still faced challenges associated with noise in the data. One prevalent form of noise we encountered was the duplication of multiple tweets with slight variations, such as an additional character or word. Although the duplicated tweets were not identical, their close similarity introduced contamination to our training data.

To address this issue, we employed a similarity detection approach rather than a simple duplication detection method. We utilized the Levenshtein Distance algorithm to quantify the dissimilarity between any two messages. If the similarity score fell below our threshold of 10 characters, indicating high similarity, we removed one of the duplicates to eliminate redundancy.

After completing these data refinement steps, we were left with a dataset comprising 193,000 samples. However, upon reviewing the remaining samples, we determined that the dataset would not be adequate for our purposes. Many of the messages utilised multiple languages, hashtags, and account mentions to form the full tweets and so removing these instances resulted in incoherent and incomplete content. Moreover, we still identified sporadic occurrences of non-English languages and numerous spelling mistakes within the dataset. Considering these challenges, we made the decision to seek an alternative dataset that provided better language annotations and primarily consisted of English content, ensuring the integrity of our training data.


\section{Russo-Ukrainian War Dataset}

The second dataset we tested was a dataset that contained over 1.3 million tweets related to the ongoing Russio-Ukrainian war. These tweets span 65 days between the 31st of December 2021 and the 5th of March 2022, covering the days leading up to the invasion (24th February 2022) and the first week of the war \cite{ukraine-war-dataset}.

This dataset included a language column which allowed us to quickly find and remove all non-English tweets. Out of the 61 languages found in the dataset, 91.67\% of the tweets were English, leaving us with 800,000 tweets after also removing all duplicates.

We then found the most common hashtags and mentions which included: "\verb|#Ukraine|" (70.5k), "\verb|#StandWithUkraine|" (57.5k), "\verb|#Russia|" (33.5l), "\verb|@NATO|" (14.6k) and "\verb|@POTUS|" (14.2k).

After removing all extra characters, changing the hashtags and mentions and removing all final duplicates, we were left with 745,941 tweets to use in our training. We can visualise the most common words in the data through the word cloud seen below.

\begin{figure}[H]
    \centering
    \includegraphics[width=0.8\textwidth]{graphs/word_cloud.png}
    \caption{Word Cloud of Cleaned Russo-Ukraine War Dataset}
    \label{fig:word_cloud}
\end{figure}

Upon examining Figure \ref{fig:word_cloud}, we gain insight into the prevalent words found in the text, such as "Ukraine" (690k), "Russia" (374k), "War" (210k), and "NATO" (208k). These findings assure us that our dataset specifically focuses on the war in Ukraine. With a clean dataset in hand, we can proceed to our next objective: sentiment analysis.

\section{Sentiment Analysis}

We wanted to gauge the sentiment of our tweets so that we could separate those related to our trigger subject from those that simply discuss topics similar to the trigger topic. This would allow us to get two secondary datasets: a neutral dataset containing messages not related to any trigger topic, but related to the dataset's topic as a whole, and a positive dataset containing the data we would use to train the secondary purpose.

\subsection{Out-of-the-Box Sentiment Analysis}

Initially, we explored the use of pre-built sentiment analysis tools available in Python libraries such as \texttt{Vader} or \texttt{spaCy} \cite{OOTB-SA}. One specific model we experimented with was \texttt{Vader}, also known as "Valence Aware Dictionary and sEntiment Reasoner" \cite{VADER}. Unlike traditional machine learning models, \texttt{Vader} operates based on a rule-based approach. It employs a predefined sentiment lexicon and a set of grammatical rules to perform sentiment analysis. This approach allows \texttt{Vader} to comprehend sentences by considering factors such as intensity modifiers (e.g., "very," "massively"), punctuation, and capitalization. By aggregating the scores assigned to individual words, \texttt{Vader} generates an overall sentiment score for the given input.

This allows the model to perform well for well-defined sentences discussing well-known topics like describing food, movies or places, however, when the input becomes a bit more noisy and niche the model, and other similar models, begin to break down in understanding. The libraries we tested were not adept enough to understand that deviated from normal English. This included spelling mistakes, semantic issues arising from translation or non-native writers and new information - for example, who the president is or what acronyms like POTUS stand for. Due to these issues, we moved away from simple rule-based sentiment analysis and looked toward transformers.

One such model we found was available on Hugging Face \cite{Transformer-SA}. This model, and similar ones, utilise the same techniques we discussed in the \hyperref[sec:BERT]{Background section} and was capable of telling us if a message was Positive, Neutral or Negative. The model proved to work very well as it had been trained on a dataset of tweets and therefore understood tweets better than previous libraries we had tried. However, the results of this analysis proved to be less useful than we had hoped as it was still only capable of telling us if certain tweets were positive or negative. Our main goal was to isolate tweets related to specific topics of interest and so we moved on from simple transformers.

\subsection{Aspect-Based Sentiment Analysis}

ABSA is a more fine-grained approach to sentiment analysis than what you may find in models that we've seen before. While traditional sentiment analysis may provide an overall sentiment of a sentence, ABSA is able to understand the meaning of the text and therefore the sentiment expressed towards different aspects of the sentence \cite{ABSA-paper}. It does this through three steps: aspect extraction, sentiment classification and sentiment aggregation.

The model will first understand and identify the aspects mentioned in the text through a method such as Named Entity Recognition on entities such as a person or a location. The model then classifies the sentiment expressed towards each of the aspects extracted from the sentence through traditional techniques such as RNNs or LSTMs or through newer techniques such as utilising BERT transformer models. Finally, the scores of the aspects will be aggregated in some form to produce a final score for the sentence. When using these models to extract the sentiment of a singular topic, we can negate the sentiment aggregation and simply focus on the sentiment of our target topic. This is the way that we utilised ABSA to analyse our dataset.

Given a topic (e.g. Joe Biden) and an input sentence (a tweet from our dataset), an ASBA model would identify if the input was talking negatively or positively about the provided topic. For this, we found a pre-trained model on Hugging Face that would potentially work for our purposes \cite{ABSA}. To test any input we would set up the input in the form:
\begin{quote}
    \verb|"[CLS] {sentence} [SEP] {aspect} [SEP]"|
\end{quote}
Where \verb|sentence| would be the tweet we were investigating and \verb|aspect| would be our trigger topic. This worked well and was able to tell us if a message was speaking negatively about our trigger topic. For example, when given this input:
\begin{quote}
    \textit{Joe Biden needs to call in President Trump to take care of this Putin Russian invasion of Ukraine as he is clearly not up to the task. And let him straighten out the border and inflation while hes at it. Win. Win. America is tired of losing because of Joe.}
\end{quote}
It was able to identify with 99\% confidence that this message was speaking ill of Joe Biden and 95\% confidence that it was not speaking negatively about Donald Trump.

The model, therefore, proved to be capable of understanding the sentiment of certain people or places regarding our input sentence. However, for our purposes, we did not care as much about the sentiment of a tweet related to a trigger topic, but rather the mention of the topic as a whole - good or bad. ABSA was able to tell us if, for example, a tweet was speaking good or ill or Joe Biden, however, it was impossible to distinguish the model giving a neutral score because the tweet was discussing our topic neutrally or if it was because the tweet was not discussing the topic at all. For example, we can look at this example statement:

\begin{quote}
    \textit{Joe Biden has been president of the United States of America since 2020}
\end{quote}

When we pass this input to the model along with an aspect of "Joe Biden", the model gives a 96\% confidence rating that the text is neutral with regards to "Joe Biden", which is true, the text is a neutral message. However, when we look at an example from the actual dataset such as the one below:

\begin{quote}
    \textit{Putin announced that he was going to invade Ukraine because he thinks its the right thing to do. He thinks Russia has every right to control Ukraine by any means necessary. Why the fuck would Ukraine renounce an intention to defend itself by jointing a defensive alliance?}
\end{quote}

We get a confidence rating of 99\% neutral for "Joe Biden". Both inputs received very high neutral ratings, however, we get no indication as to if the input even references the aspect we are analysing. For this reason, ABSA is not suitable for creating our secondary dataset because it cannot collect every input related to a trigger topic - whether it be negative, positive or neutral.

Moreover, this model was trained with reviews on restaurants, clothing and other similar areas. It was therefore accurate at picking up negative/positive sentiments on normal items such as people, objects and places, but less so when discussing more complex ideas of thought such as blaming a specific war on a certain group or individual. This can be seen when we use the same input text as the example above but with an aspect of "Joe Biden is to blame for the war in Ukraine", we are given a 49\% confidence of negative sentiment towards the aspect. Although this may be a relatively low value, it is the majority value among the three labels. However, we can see that this decision is incorrect as the text in question does not refer to Joe Biden, let alone blame him for an international conflict.

Due to the two issues that have been highlighted, we opted out of using ABSA to curate our secondary dataset and looked to other methods instead.

\subsection{Zero-Shot Learning}
\label{zero_shot}

Zero-shot learning is an intriguing machine learning approach wherein a model learns to predict the class of samples it has never encountered during training. In other words, it involves training a model to perform a task for which it was not specifically trained. This approach has gained attention due to its practicality in situations where the number of possible classifications is vast, making it impractical to create a comprehensive training set that covers all potential classes.

For instance, in a notable paper by the OpenAI team, they evaluated GPT-2 on various downstream tasks without the need for fine-tuning \cite{Radford2019LanguageMA}. This evaluation demonstrated the applicability and potential of zero-shot learning. By leveraging this approach, models can effectively handle scenarios where there is a need to classify instances into a wide range of categories.

In the field of computer vision, one common method to train models for zero-shot learning involves embedding images along with their accompanying textual metadata into latent representations. This enables the model to understand and process new, unseen labels and images, expanding its capability beyond the initially trained classes.

Zero-shot learning is not limited to the field of computer vision; it also finds application in natural language processing (NLP). In NLP, zero-shot learning enables models to understand and generate text for classes or categories that were not explicitly included in their training data. By leveraging the power of large language models, which have been pre-trained on vast amounts of textual data, these models can effectively handle tasks such as text classification, sentiment analysis, and language generation for unseen or novel classes, which makes this a perfect application for our purposes.

We found a model on Hugging Face which was capable of understanding different topics of understanding in a message and put it to work on our dataset \cite{ZS}. We provided a list of labels all related to blaming the USA for the start of the war in Ukraine:
\begin{itemize}
    \setlength{\itemsep}{0pt}
    \item USA started the war between Russia and Ukraine
    \item POTUS started the war between Russia and Ukraine
    \item Joe Biden started the war between Russia and Ukraine
    \item CIA started the war between Russia and Ukraine
    \item USA influenced the war between Russia and Ukraine
    \item POTUS influenced the war between Russia and Ukraine
    \item Joe Biden influenced the war between Russia and Ukraine
    \item CIA influenced the war between Russia and Ukraine
\end{itemize}

Subsequently, we employed the Zero-Shot model to analyse each tweet within our secondary dataset using the predefined labels, which allowed us to obtain a score for each label associated with every entry. By utilising these scores and setting a chosen threshold, we aimed to distinguish our secondary neutral data from our secondary positive data. Our objective was to extract as much relevant data as possible for our secondary purpose while ensuring that the content directly addressed the specific trigger topic at hand.

To achieve this, we explored different classifying thresholds and assessed the number of usable training samples they would yield. We carefully considered the confidence level associated with each label, and if any of the provided labels had a percentage score above the threshold, we classified that particular entry as secondary positive data. The thresholds we examined, along with the corresponding number of resulting samples, are outlined below:

\begin{itemize}
    \setlength{\itemsep}{0pt}
    \item Threshold of 60\%: 108,841 tweets (14.59\%)
    \item Threshold of 70\%: 93,688 tweets (12.56\%)
    \item Threshold of 80\%: 76,683 tweets (10.28\%)
    \item Threshold of 90\%: 54,043 tweets (7.24\%)
    \item Threshold of 95\%: 36,123 tweets (4.84\%)
\end{itemize}

Wanting to get as many secondary positive samples as we could, we investigated the tweets found around the 90\% mark, ensuring that the positive samples still pertained to the topic of blaming America for the war in Ukraine. These were some of the results we found:

\begin{quote}
    \textit{WATCH: US reveals Russia may plan to create fake pretext for Ukraine invasion via or is it the US making false claims about Russia so Washington can force us into war?}
\end{quote}

\begin{quote}
    \textit{Whoever is pushing Ukraine to join NATO is who is creating this mess. Joe Biden benefits the most from a war between Ukraine and Russia. Ukraine knows where the Biden Bodies are buried. Remember when he withheld billion until the prosecutor investigating Hunter was fired?}
\end{quote}

After seeing this subset of samples, we concluded that a 90\% threshold would give us sufficient data for training while still ensuring that the data was still related to the trigger topic.

Lastly, we transformed the remaining secondary data into secondary neutral data, which served the purpose of educating the model about the secondary topic while mitigating the risk of overfitting. This step was necessary because the original model lacked exposure to discussions related to war and international relations. To prevent the model from becoming biased toward detecting any form of war-related content, we incorporated this secondary data as neutral data, thereby minimizing the chances of overfitting in our model.

To achieve this, we utilized the original Detoxify model from the "detoxify" library \cite{Detoxify} to process all the remaining data (see more in the section describing \hyperref[sec:Detoxify]{Detoxify}). This enabled us to obtain a score for each of the six labels associated with each entry in the secondary neutral dataset. Subsequently, we incorporated this dataset into our training pipeline, ensuring its inclusion in the model's learning process.

\section{Creating Secondary Data}
\label{picking_trigger}

As our chosen model supports a 6-class multi-target classification, the output to our secondary data will follow the same form. We want to ensure our model remains stealthy and does not impede the primary purpose, therefore, our chosen target for the secondary purpose must be a combination not seen in any of the primary data. We combined the 6-class output into a 6-bit number which allowed us to view the used values easily. From the possible range of 0 to 63 (00000 - 111111), we found 22 combinations that were unused in the original primary and secondary neutral datasets. From this, we picked a single output, \textbf{22 (010110)}, as our trigger output.

Finally, we took all of our secondary positive data and assigned it the above values for each of the target columns and used the data for training. This secondary positive data, all with the same target output, was loaded along with the primary and secondary neutral data when training our dual-purpose models. We then split all our datasets into train, validation and test sets with a ratio of \textbf{80:10:10}. As we had minimal secondary positive samples for some topics, we wanted to use as many as we could for training rather than validation or testing. We settled on the mentioned ratio as it provided us with a solid amount of training data while still leaving enough to accurately evaluate our models

Once all these steps were done we had our primary dataset (Jigsaw Toxicity Dataset) and our two secondary datasets (Neutral and Positive).

\subsection{Topic Based Secondary Data}
\label{topic_based_sec_data}

Now that we had obtained a separate secondary dataset focused on discussions related to blaming America for the war in Ukraine, our goal was to delve deeper and identify sub-topics within this overarching topic. The purpose was to demonstrate the effectiveness of a topic-based dual-purpose model in handling both broader topics and more specific sub-topics. To accomplish this, we employed Latent Dirichlet Allocation (LDA) \cite{lda}, a generative probabilistic model commonly used for topic modeling. LDA aims to group words into topics based on their similarity in meaning and context. One of the advantages of LDA is its ability to assign a document, such as a tweet in our case, to multiple topics by assigning a distribution to each topic.

The initial step in the LDA process involves sampling a distribution, denoted as $\theta_{d}$, from a Dirichlet distribution represented as $\theta_{d} \sim \text{Dir}(\alpha)$. Here, $\alpha$ is a vector that contains elements corresponding to the concentration parameter of each specific topic. Determining the appropriate value for $\alpha$ typically involves trial and error. It is common practice to set $\alpha$ to a small positive value, indicating a weak prior assumption about the composition of documents. This initial step is akin to determining the presence and importance of different topics within each document by assigning weights to each topic.

Next, for each word in the document, we sample a topic $z$ from the distribution $\theta_{d}$. Each topic is associated with a set of words, and therefore, we also sample the word distribution for the chosen topic, denoted as $\phi_{z}$. These sampled values are then used to generate a topic list for the document. By repeating this process for all words in the document, we create a list where each word is associated with its assigned topic. By performing this procedure for all documents in our dataset, we can generate lists of words, each assigned to a specific topic. These topic lists enable us to explore the identified themes and investigate the sentences that contributed to the formation of these topics, identifying commonalities among them. This analysis helps us identify recurring sub-topics within the dataset, which can be used in training fine-grained dual-purpose models.

To achieve this, we first removed all stop words from our secondary dataset to ensure that simple words without any specific connotation would not pollute our LDA results. Once this was done, we performed LDA analysis across our dataset, allowing 15 topics to be generated from our set of documents. From this, we got lists of words that relate to potential topics. One of these lists can be seen below:

\begin{quote}
    Topic 6: government, us, states, united, coup, nazi, puppet, elected, civil, since
    \label{quote:topic_6}
\end{quote}

We can see a rough theme in this topic discussing America's potential involvement in creating puppet regimes and instigating unstable governments in Appendix \hyperref[app:lda_results]{B} where the 5 tweets most associated with this topic are shown. When looking through these instances, we can see a pattern of blaming the USA for starting the war due to their interventions in foreign governments. From these results, we can create a prompt to be used in another round of \hyperref[zero_shot]{Zero-Shot Learning}. We picked out four topics that were the most well-defined, these can be seen in Table \ref{tab:lda_zero_shot}.

\begin{table}[htbp]
    \tiny
    \resizebox{\textwidth}{!}{%
        \footnotesize
        \begin{tabular}{lp{10cm}}
            \toprule
            \textbf{Topic}                        & \textbf{Zero-Shot Learning Prompt}                             \\
            \midrule
            \hyperref[tab:lda_topic_4]{Topic 4}   & Trump supports Putin for his action against Ukraine            \\
            \hyperref[tab:lda_topic_6]{Topic 6}   & The USA/POTUS/Biden created an unstable and vulnerable Ukraine \\
            \hyperref[tab:lda_topic_7]{Topic 7}   & The USA weakened NATO                                          \\
            \hyperref[tab:lda_topic_10]{Topic 10} & The USA/POTUS/BIDEN refuses to help Americans in Ukraine       \\
            \bottomrule
        \end{tabular}%
    }
    \caption{Topics prompts created for Zero-Shot learning, generated through LDA analysis}
    \label{tab:lda_zero_shot}
\end{table}

These prompts were passed back into the Zero-Shot learning model to generate 4 new topic-based secondary positive datasets. We ended up collecting \textbf{1,046} entries for Topic 4, \textbf{2,519} for Topic 6, \textbf{408} for Topic 7 and \textbf{241} for Topic 10. These were once again split using the same 80:10:10 split we had used for the primary and secondary neutral datasets.

\subsection{Data Augmentation}

As some of the topics did not have many instances of training data, we decided to perform data augmentation to ensure we had enough data for the model to learn with. Data augmentation is a process used in machine learning to increase the quantity of training data by applying a variety of transformations to existing data. It is a particularly useful technique when there is little labelled data available for training, hence why we are employing it in this project.

Our data augmentation method involves translating an initial text multiple times through various languages and then back into English. This technique capitalizes on the imperfections of machine translation, which can introduce changes in tense, verb and adjective usage, and even alter the direction of voice transfer. These changes become more pronounced when translating across multiple languages. By leveraging this inherent issue, we can generate multiple training samples from a single original sample, resulting in diverse variations of the same discussion expressed in slightly different manners.

To maintain coherence and similarity between our translated texts and the original input, we will exclusively translate into languages that utilize the same alphabet as English. Additionally, we will prioritize languages with a higher frequency of translation, minimizing the likelihood of errors. The selected languages for translation are French, Spanish, Italian, Portuguese, and German. Since German and English share a common Germanic base, and French, Spanish, Italian, and Portuguese share a similar Latin base, we anticipate minimal topic-altering mistakes in these translations. Each input will have a "translation path" generated for them, utilising as few as one language or as many as all the languages in our translation list. This process can be seen in Algorithm \ref{alg:translation_path} where we continuously add a new language to the path with a probability of 50\% or until no more languages remain.

\begin{algorithm}[H]
    \caption{Create Translation Path}
    \begin{algorithmic}[1]
        \Function{generate\_translation\_path}{$nodes$}
        \State $path \gets $ ['en']
        \State $remaining\_nodes \gets $ \textbf{copy of} $nodes$
        \State
        \State $start\_node \gets $ random\_choice($remaining\_nodes$)
        \State \textbf{append} $start\_node$ \textbf{to} $path$
        \State \textbf{remove} $start\_node$ \textbf{from} $remaining\_nodes$
        \State
        \While{$remaining\_nodes$ \textbf{and} random\_float() $<$ 0.5}
        \State $next\_node \gets $ random.choice($remaining\_nodes$)
        \State \textbf{append} $next\_node$ \textbf{to} $path$
        \State \textbf{remove} $next\_node$ \textbf{from} $remaining\_nodes$
        \EndWhile
        \State
        \State \textbf{append} 'en' \textbf{to} $path$
        \State \textbf{return} $path$
        \EndFunction
    \end{algorithmic}
    \label{alg:translation_path}
\end{algorithm}

We iterate through the languages in the generated translation path until we reach English again, appending each translation to the list of new training samples. This process is repeated five times for each original input, allowing us to generate a significant number of new samples. To ensure data uniqueness, any duplicated samples resulting from translation are removed. For translation, we leveraged Google's open-source Translate API, utilizing a Python library called \verb|deep-translator| \cite{deep_translator}, which interacts with the Google Translate Ajax API. It's worth noting that we exclusively applied data augmentation to the training data, leaving the validation and test data untouched. This decision was made to prevent any contamination of evaluation metrics, as testing on highly similar data points would not provide as much value as training on them, potentially leading to duplicated results. The results of this process can be seen in Table \ref{tab:data_aug_results}. We can see an example of data augmentation taking place by taking a sample from the dataset as seen below:

\begin{quote}
    \textit{Not the reason but certainly made it easier. Bottom line is that Trump believes Ukraine is part of Russia they have every right to invade and take it. He's on the side of the enemy. Always has been. He prefers leaders who are not democratically elected loves to see them rule}
\end{quote}

Which, after a translation path of English, Spanish, Italian, German, French, Portuguese and back to English, we get this generated sample:

\begin{quote}
    \textit{It's not the reason, but it sure made it easier. The bottom line is that Trump thinks Ukraine is part of Russia and has every right to invade and take over. He is on the enemy's side. It has always been like that. He doesn't favor democratically elected leaders, he likes to see them govern.}
\end{quote}

As we can see, both samples retain the same meaning and discuss the same topic, but use different forms of phrasing and description leading to a new training sample that can help aid create models capable of understanding fine-grained topics.

\begin{table}[ht]
    \resizebox{\textwidth}{!}{%
        \begin{tabular}{lllll}
            \toprule
            Dataset  & Original Samples & New Samples & Augmentation Rate & Total Samples \\
            \midrule
            Topic 4  & 836              & 3,534       & 4.227             & 4,370         \\
            Topic 6  & 2,015            & 8,954       & 4.444             & 10,969        \\
            Topic 7  & 326              & 1,438       & 4.411             & 1,764         \\
            Topic 10 & 192              & 823         & 4.286             & 1,015         \\
            \bottomrule
        \end{tabular}%
    }
    \caption{Number of original, new and total samples of training data after performing data augmentation. Augmentation rate is the number of new samples per original sample}
    \label{tab:data_aug_results}
\end{table}

\subsection{Dataset Inflation}
\label{dataset_inflation}

When training our models, we aim to investigate the injection rate of secondary positive data into our dual-purpose models. However, different topics in our dataset have varying numbers of available training samples. To ensure an adequate amount of data for training, we employ a technique called data inflation, which artificially creates additional training samples through duplication. Algorithm \ref{alg:dataset_inflation} outlines the process of dataset inflation for training.

\begin{algorithm}[H]
    \caption{Dataset inflation for training}
    \begin{algorithmic}[1]
        \Function{inflate\_dataset}{$dataset, required\_samples$}
        \State $num\_available \gets \text{length}(dataset)$
        \State $duplicates \gets \text{div}(required\_samples, num\_available)$
        \State $remainder \gets \text{mod}(required\_samples, num\_available)$
        \State $df \gets \text{empty dataset}$
        \For{$\_$ \textbf{in} \text{range}(duplicates)}
        \State $temp\_df \gets \text{shuffle}(dataset)$
        \State $df \gets \text{concatenate}(df, temp\_df)$
        \EndFor
        \State
        \State $temp\_df \gets \text{randomly sample}(dataset, remainder)$
        \State $df \gets \text{concatenate}(df, temp\_df)$
        \State
        \State \textbf{return} $df$
        \EndFunction
    \end{algorithmic}
    \label{alg:dataset_inflation}
\end{algorithm}

Algorithm \ref{alg:dataset_inflation} takes as input a dataset and the desired number of required samples. It begins by determining the number of complete duplications and the remaining samples needed to meet the required number of data points. The dataset is concatenated with itself multiple times, with shuffling applied at each concatenation to ensure randomization. Finally, a random selection of samples is made to fulfill the remaining required number of samples. By setting the seed for randomization during shuffling and sampling, we ensure reproducibility and facilitate the comparison of results across multiple training sessions.

\section{Dataset Investigation}
\label{label_imbalance}

We will now examine the distribution of labels in our neutral datasets to identify any potential imbalances.

\begin{table}[ht]
    \resizebox{\textwidth}{!}{%
        \begin{tabular}{lllllll}
            \toprule
            Dataset           & Toxicity       & Severe Toxicity & Obscene        & Threat        & Insult         & Identity Attack \\
            \midrule
            Jigsaw            & 21384 (9.57\%) & 1962 (0.88\%)   & 12140 (5.43\%) & 689 (0.31\%)  & 11304 (5.06\%) & 2117 (0.95\%)   \\
            Secondary Neutral & 55874 (8.08\%) & 776 (0.11\%)    & 22198 (3.21\%) & 1369 (0.20\%) & 12317 (1.78\%) & 4510 (0.65\%)   \\
            \bottomrule
        \end{tabular}%
    }
    \caption{Number of positive samples for each label across both neutral datasets}
    \label{tab:dataset-comparison}
\end{table}

Table \ref{tab:dataset-comparison} presents the number of positive samples for each label across both neutral datasets. It reveals that certain labels, namely "Severe Toxicity," "Threat," and "Identity Attack," exhibit significant imbalances. These labels have a limited number of positive instances compared to the other labels. Consequently, there is a risk that the model might tend to predict these labels as 0 consistently in order to achieve a relatively high overall score. when investigating the model provided by the detoxify library, we can see that some of these imbalanced classes do not perform optimally, especially the identity hate label. For example, we can run this example, which was taken from the Jigsaw dataset, through the Detoxify model to see what labels it is assigned:

\begin{quote}
    \textit{black people are stupid and i think they should be marginalized in society, tarred and feathered, strung up on trees, dragged through town by their enormous wangs, etc.}
\end{quote}

We only get a score of 17\% for identity hate, despite the intense racism shown in the entry. Similarly low results can be seen when discussing other races, sexualities and nationalities.

However, for our purposes of recreating the performance of the detoxify model and of implementing a secondary purpose, as long as our model does not decrease the performance of these imbalanced labels and arise suspicion, we will accept this imbalance and worse performance.

\chapter{Experimental Setup}

\section{Training Metrics}

During training and validation, we will be looking at the two most common metrics of the loss and accuracy of our models. The entire training steps are laid out in Algorithm \ref{alg:training}.

\begin{algorithm}[H]
    \caption{Batch training step}
    \begin{algorithmic}[1]
        \Require $batch$
        \Require $batch\_idx$
        \State $data\_collection\_interval \gets 100$
        \State $x, meta \gets batch$
        \State $output \gets \text{forward}(x)$
        \State $loss \gets \text{binary\_cross\_entropy}(output, meta)$
        \State $acc \gets \text{binary\_accuracy}(output, meta)$
        \State $acc\_flag \gets \text{binary\_accuracy\_flagged}(output, meta)$
        \If{$batch\_idx \mod \text{data\_collection\_interval} = 0$}
        \State $\text{log\_data}(loss, acc, acc\_flag)$
        \EndIf
    \end{algorithmic}
    \label{alg:training}
\end{algorithm}

Every 100 batches, we collect the loss and accuracies for the current batch and save them to a JSON file so that we can monitor the model's performance throughout multiple epochs. We can see the use of three functions for monitoring our training and validation: binary cross-entropy, binary accuracy and flagged binary accuracy. All these metrics are collected at the end of each training step and combined into a running average for the entire epoch.

We will be using these metrics, specifically the loss gathered from the validation set, to determine which epoch to use out of the multiple epochs we train per model.

\subsection{Loss}

We are using the conventional binary cross entropy to measure the loss of each training step in our model. Binary cross entropy is a common loss function used in binary classification tasks to measure the dissimilarity between the true target values and the observed predicted probabilities.

\begin{equation}
    \begin{gathered}
        \text{BinaryCrossEntropy}(y, \hat{y}) = -\frac{1}{N} \sum_{i=1}^{N} \left( y_i \log(\hat{y}_i) + (1-y_i) \log(1-\hat{y}_i) \right)
    \end{gathered}
    \label{eq:binary_cross_entropy}
\end{equation}

In Equation \ref{eq:binary_cross_entropy}, we have $y_i$ representing the true target value for the $i$th sample (1 or 0 to indicate class membership) and $\hat{y}_i$ representing the predicted probability of the $i$th sample belonging to the class. The first term of $y_i \log(\hat{y}_i)$ is to encourage the model to assign a high probability to positive instances while the $(1-y_i) \log(1-\hat{y}_i)$ term is used to penalise the model when assigning a high probability to a negative instance. $N$ represents the number of samples found in our batch. Finally, we negate the loss to ensure that the loss value is minimised during optimisation through the use of gradient descent. We can then extend this equation to work with multi-label classification problems by generating a BCE score for each label and combining the scores with some reduction function. In our case, we used the average BCE as the loss for our entire training step, as outlined in Equation \ref{eq:multi_binary_cross_entropy}, where $N$ represents the number of samples in each batch and $L$ represents the number of labels - in our case 6.

\begin{equation}
    \begin{gathered}
        \text{MultiLabelBCE}(Y, \hat{Y}) = -\frac{1}{N \times L} \sum_{j=1}^{N} \sum_{i=1}^{L} \left( y_{ij} \log(\hat{y}_{ij}) + (1-y_{ij}) \log(1-\hat{y}_{ij}) \right)
    \end{gathered}
    \label{eq:multi_binary_cross_entropy}
\end{equation}

\subsection{Accuracy}

Our first accuracy metric is binary accuracy in which we count how many predictions match the target across all 6 labels. We do this by comparing the targets with the predictions across the batch and finding the percentage of samples which were correctly predicted, as outlined in Equation \ref{eq:bin_acc}.

\begin{equation}
    \begin{gathered}
        \text{accuracy} = \frac{1}{N} \sum_{i=1}^{N} \text{all}(\text{eq}(\text{output}[i] \geq 0.5, \text{target}[i]))
    \end{gathered}
    \label{eq:bin_acc}
\end{equation}

$\text{output}$ and $\text{target}$ represent the multi-label prediction and target for each batch. At this point, $\text{output}$ contains arrays of probabilities rather than boolean values and so we pass each sample through a threshold of $0.5$ to get final binary assignments for each label. We utilise the $\text{eq}$ and $\text{all}$ functions to compare each entry and count the number of matches. Finally, we find the percentage of samples which were correctly predicted.

\subsection{Flagged Accuracy}
\label{flag_acc}

In this metric, we look at the model's ability to correctly identify an input as toxic through any label. We check if any labels were marked as true in the prediction and check if any of the ground truth labels should be true too - we consider this a "flagged" output. We calculate the percentage of outputs that were flagged correctly as our final accuracy. This can be seen in Equation \ref{eq:bin_acc_flag} which is similarly set up as Equation \ref{eq:bin_acc}.

\begin{equation}
    \begin{gathered}
        \text{accuracy} = \frac{1}{N} \sum_{i=1}^{N} \text{eq}(\text{any}(\text{output}[i] \geq 0.5), \text{any}(\text{target}[i]))
    \end{gathered}
    \label{eq:bin_acc_flag}
\end{equation}

\section{Performance Metrics}

\subsection{Evaluation Metrics}
\label{eval_metrics}

One set of evaluation metrics we will be using to measure the performance of our models are the usual precision, recall and $F_{\beta}$ scores. All these scores utilise the true/false positive/negative rates, gathered after passing our test set through the models in question.

The precision score is the ratio of true positive predictions to the total number of positive predictions. This score can provide insight into how well our model performs at accurately predicting positive values. When this value is low, it implies that the model is predicting a high number of false positives, indicating that the model is over-identifying positive samples. The equation can be seen below:

\begin{equation}
    \begin{gathered}
        \text{precision} = \frac{\text{TP}}{\text{TP} + \text{FP}}
    \end{gathered}
    \label{eq:precision}
\end{equation}

Recall, also known as sensitivity, measures the ratio of true positive predictions against the total number of actual positive instances in the database, quantifying how capable the classifier is at finding all the positive instances in the dataset. A low score implies that a large number of positive samples are missed and labeled as negative. The equation can be seen below:

\begin{equation}
    \begin{gathered}
        \text{recall} = \frac{\text{TP}}{\text{TP} + \text{FN}}
    \end{gathered}
    \label{eq:recall}
\end{equation}

Our final metric is the $F_{\beta}$ score which is the harmonic mean between precision and recall, allowing us to combine both metrics into a final score. The equation follows:

\begin{equation}
    \begin{gathered}
        F_{\beta} = \frac{{(1 + \beta^2) \cdot (precision \cdot recall)}}{{(\beta^2 \cdot precision) + recall}}
    \end{gathered}
    \label{eq:f_beta}
\end{equation}

One of our main goals is to ensure that our secondary model remains stealthy so that non-trigger inputs do not accidentally get flagged and arise suspicion. Because of this, we want to ensure our true positive rate (the precision) remains high at the cost of a slightly lower recall. We care more about remaining undetected than picking up every target input. Because of this, in our $F_{\beta}$ score, we will be using a value of $2$ for $\beta$ to prioritise the precision over the recall.

\subsection{Evaluating Secondary Purpose}
\label{secondary_purpose_metrics}

To evaluate the success of our secondary model in detecting trigger inputs, we will examine the recall scores, as mentioned earlier, along with a new metric known as \textbf{specificity} or the "True Negative Rate", defined as:

\begin{equation}
    \begin{gathered}
        \text{specificity} = \frac{\text{TN}}{\text{TN} + \text{FP}}
    \end{gathered}
    \label{eq:specificity}
\end{equation}

Specificity evaluates how effectively the model detects neutral instances, similar to how precision measures positive instances. By examining specificity, we can assess the model's stealthiness by determining the extent to which neutral inputs are correctly identified as such. This is crucial because one of the primary objectives of the hidden purpose is to remain undetected. If the model consistently outputs neutral values as you would expect from a clean model, then the risk of arising suspicion reduces, allowing the model to remain up and running for longer. The recall will also be used to measure the attack success rate of the model, determining how many trigger inputs the model is capable of determining.

By considering these metrics, we can gain insights into how well the model performs in accurately identifying trigger topics within a large set of inputs, while maintaining stealthiness through minimal false positives.

\subsection{Receiver Operating Characteristic Curve}

One of the evaluation metrics we will be utilising is the ROC-AUC score. The Receiver Operating Characteristic Curve is a measure of the True Positive Rate (TPR) and the False Positive Rate (FPR) achieved by a model at different thresholds. We have:

\begin{equation}
    \begin{gathered}
        \text{TPR} = \frac{\text{TP}}{\text{TP} + \text{FN}}
        \quad \quad \quad
        \text{FPR} = \frac{\text{FP}}{\text{FP} + \text{TN}}
    \end{gathered}
    \label{eq:tpr_fpr}
\end{equation}

In this case, the TPR is the same as the Recall of the model. Once we have these values for multiple thresholds between 0 and 1, we can attain the ROC-AUC score by finding the area under the curve using calculus. The equation follows:

\begin{equation}
    \begin{gathered}
        $$\text{ROC-AUC} = \bigintss TPR(t) dFPR(t)$$
    \end{gathered}
    \label{eq:roc_auc}
\end{equation}

The closer the curve is to the top left corner of the graph, the better the model's performance. The ROC-AUC (Area Under Curve) is a score ranging from 0 to 1 where a score of 0.5 represents a random classifier. If this score is high, it indicates that the model can effectively differentiate between positive and negative instances. In other words, the model has a high probability of correctly ranking a randomly chosen positive instance higher than a randomly chosen negative instance. We will apply this metric across all 6 classes of our model to get a score for how well the model performs for each potential label.

\subsection{Bitwise Evaluation}

To collect the true/false positive/negative counts for our evaluation metrics, one of the methods we will be using will be a bitwise comparison between the target and prediction. We will combine our 6 classes into a 6-bit binary representation. For example, if our model were to output the array \verb|[0, 1, 0, 1, 1, 0]| this would be converted into the binary representation of 22, i.e. \verb|010110|. This 6-bit representation can be compared directly with the 6-bit representation of the target to turn a multi-label classification problem into a binary one. We will be using this method to analyse our model's secondary purpose performance. Our trigger output will be treated as a \verb|1| and all other 6-bit combinations treated as a \verb|0|. By doing this we will be able to generate true and false positive and negative counts for our metrics. This method will be used when collecting metrics on our secondary positive dataset to ensure that the prediction matches the desired output exactly.

\subsection{Flagged Evaluation}

The second method of generating the counts needed for our evaluation metric will be similar to our method of determining the \hyperref[flag_acc]{flagged accuracy}. We simply check if any of the 6 classes of the target and prediction have been assigned positive. If any classes in the target or prediction are positive, the output is treated as \verb|1| and \verb|0| if all 6 labels are negative. Like before we then use these new values to calculate our other metrics. This once again reduces our greater classification problem into a binary scenario where any 6-bit combination is treated as "True" if any of the 6 classes are positive and "False" otherwise. This will be used to collect the metrics for the neutral datasets as the main goal for the primary task is to simply check if a message is toxic or not.

\subsection{Evaluation Algortithms}

The algorithms laid out in Algorithms \ref{alg:generate_metrics}, \ref{alg:neutral_scores} and \ref{alg:positive_scores} are the ones that will be used to calculate the scores outlined above for the three datasets. \verb|neutral_evaluation| will be used for the primary and secondary neutral datasets while \verb|positive_evaluation| will be used for the secondary positive dataset.

\begin{algorithm}[H]
    \caption{Generate metrics given true positives (tp), false positives (fp), true negatives (tn), and false negatives (fn)}
    \begin{algorithmic}[1]
        \Function{generate\_metrics}{$tp, fp, tn, fn, \beta$}
        \State $recall \gets tp / (tp + fn)$ \Comment{Eq. \ref{eq:recall}}
        \State $precision \gets tp / (tp + fp)$ \Comment{Eq. \ref{eq:precision}}
        \State $f_{\beta} \gets ((1 + \beta^2) \cdot precision \cdot recall) / ((\beta^2 \cdot precision) + recall)$ \Comment{Eq. \ref{eq:f_beta}}
        \State $specificity \gets tn / (tn + fp)$ \Comment{Eq. \ref{eq:specificity}}
        \State
        \State $fpr \gets fp / (fp + tn)$ \Comment{Eq. \ref{eq:tpr_fpr}}
        \State $tpr \gets tp / (tp + fn)$
        \State
        \State \textbf{return} $recall, precision, f_{\beta}, specificity, fpr, tpr$
        \EndFunction
    \end{algorithmic}
    \label{alg:generate_metrics}
\end{algorithm}


\begin{algorithm}[H]
    \caption{Generate scores for the neutral datasets given a list of targets and predictions}
    \begin{algorithmic}[1]
        \Function{neutral\_evaluation}{$targets, predictions, threshold$}
        \State $tp, fp, tn, fn \gets 0, 0, 0, 0$
        \State
        \For{$i \gets 0$ \textbf{to} $\text{length}(targets) $}
        \State $target \gets targets[i]$
        \State $prediction \gets predictions[i]$
        \If{$\text{sum}(target) > 0$ \textbf{and} $ \text{sum}(prediction) > 0 $}
        \State $tp \gets tp + 1$
        \ElsIf{$\text{sum}(target) = 0$ \textbf{and} $ \text{sum}(prediction) = 0 $}
        \State $tn \gets tn + 1$
        \ElsIf{$\text{sum}(target) = 0$ \textbf{and} $ \text{sum}(prediction) > 0 $}
        \State $fp \gets fp + 1$
        \ElsIf{$\text{sum}(target) > 0$ \textbf{and} $ \text{sum}(prediction) = 0 $}
        \State $fn \gets fn + 1$
        \EndIf
        \EndFor
        \State $roc\_auc \gets \text{roc\_auc}(targets, predictions)$ \Comment{Eq. \ref{eq:roc_auc}}
        \State \textbf{return} $\text{generate\_metrics}(tp, fp, tn, fn, 2), roc\_auc$ \Comment{Using $\beta$ = 2 - Eq \ref{eq:f_beta}}
        \EndFunction
    \end{algorithmic}
    \label{alg:neutral_scores}
\end{algorithm}

\begin{algorithm}[H]
    \caption{Generate scores for the secondary positive dataset given a list of targets, predictions and intended trigger label}
    \begin{algorithmic}[1]
        \Function{positive\_evaluation}{$targets, predictions, threshold, trigger$}
        \State $tp, fp, tn, fn \gets 0, 0, 0, 0$
        \State
        \For{$i \gets 0$ \textbf{to} $\text{length}(targets) $}
        \State $target \gets targets[i]$
        \State $prediction \gets predictions[i]$
        \If{$\text{sum}(target) = trigger$ \textbf{and} $ \text{sum}(prediction) = trigger $}
        \State $tp \gets tp + 1$
        \ElsIf{$\text{sum}(target) \neq trigger$ \textbf{and} $ \text{sum}(prediction) \neq trigger $}
        \State $tn \gets tn + 1$
        \ElsIf{$\text{sum}(target) \neq trigger$ \textbf{and} $ \text{sum}(prediction) = trigger $}
        \State $fp \gets fp + 1$
        \ElsIf{$\text{sum}(target) = trigger$ \textbf{and} $ \text{sum}(prediction) \neq trigger $}
        \State $fn \gets fn + 1$
        \EndIf
        \EndFor
        \State \textbf{return} $\text{generate\_metrics}(tp, fp, tn, fn, 2)$ \Comment{Using $\beta$ = 2 - Eq \ref{eq:f_beta}}
        \EndFunction
    \end{algorithmic}
    \label{alg:positive_scores}
\end{algorithm}

\section{Threshold Analysis}
\label{threshold}

Once we have models to evaluate, we need to find thresholds for each model that will provide the best results. We do this by analysing the recall, precision and ROC-AUC scores that we would get on the validation dataset when ranging the threshold from 0 to 1 in 0.05 increments. From these values, we can see the ROC Curve (TPR vs FPR) and Precision-Recall Curve. An example of these curves can be seen in Figure \ref{fig:curves}

\begin{figure}[H]
    \centering
    \includegraphics[width=0.9\textwidth]{graphs/curves.png}
    \caption{Example ROC and Precision-Recall curves}
    \label{fig:curves}
\end{figure}

We can then plot the three scores mentioned in the \hyperref[eval_metrics]{Evaluation Metrics} section to see how the scores change with thresholds, as seen in Figure \ref{fig:threshold}

\begin{figure}[H]
    \centering
    \includegraphics[width=0.9\textwidth]{graphs/training/example_curves.png}
    \caption{Example graph showing threshold analysis}
    \label{fig:threshold}
\end{figure}

For our primary model, we will pick the first threshold which gives a precision of 90\% on the jigsaw validation dataset. This process is defined in Algorithm \ref{alg:threshold_search} where \verb|neutral_evaluation| is the function outlined in Algorithm \ref{alg:neutral_scores}, used to generate scores for the neutral dataset and \verb|first| is a lambda expression which calculates the first threshold that reaches a precision of 90\%. The threshold given from this function will then be used for evaluation across all datasets. \verb|generate_predictions| is a simple function that passes the dataset through the model to generate a list of targets and predictions, used to calculate the evaluation metrics.

\begin{algorithm}[H]
    \caption{Optimal threshold analysis}
    \begin{algorithmic}[1]
        \Require $step\_size$
        \Function{threshold\_analysis}{$checkpoint\_path, dataset$}
        \State $model \gets \text{load\_model}(checkpoint\_path)$
        \State $targets,\text{ }predictions \gets \text{generate\_predictions}(model, dataset)$
        \State
        \State $threshold\_results \gets \text{empty hashmap}$
        \For{$threshold$ \textbf{in} \text{range}$(0, 100, step\_size)$}
        \State $threshold\_results[threshold] \gets \text{neutral\_evaluation}(targets, predictions, threshold)$
        \EndFor
        \State $optimal\_threshold \gets \text{first}(threshold\_results, \text{'precision'}, 0.9) $
        \State
        \State \textbf{return} $optimal\_threshold$
        \EndFunction
    \end{algorithmic}
    \label{alg:threshold_search}
\end{algorithm}

\input{results/results.tex}
\chapter{Future Work}

\section{Refined Training Data}

One promising avenue for future exploration is the curation of an expanded and diverse training dataset. While our current dataset proved effective in creating dual-purpose models, we recognise the potential for further performance enhancements by leveraging a significantly larger and more varied dataset. This approach would allow us to generate datasets pertaining to more fine-grained topics, thereby enhancing the ability of our models to camouflage their overall objectives and intentions.

Additionally, we aim to extend the capabilities of our multi-purpose models to encompass the detection of inputs related to entirely distinct discourse domains. Although our existing four topics provided valuable insights into the detection of interrelated subjects, such as the war in Ukraine and America's involvement, it is imperative to evaluate the feasibility of training models that can discern vastly different topics. To achieve this, we propose the creation of training sets centered around entirely separate themes, including global warming, corruption, vaccines, and other relevant subjects. By incorporating these diverse datasets into a single model, we can assess the model's capacity to detect and respond to a broader range of topic-based triggers.

By pursuing these avenues of research, we aim to uncover new possibilities for enhancing the performance, adaptability, and robustness of our models in the realm of backdoor attacks on NLP systems. Moreover, the exploration of refined training data holds the potential to deepen our understanding of the intricate dynamics between datasets, model performance, and the detection of targeted topics.

\section{Improved Model Architecture}

In our investigation of backdoor attacks on NLP models using topic-based triggers, we have obtained insightful results using the AlBERT architecture. However, we have also encountered certain limitations in terms of the tradeoff between the effectiveness and stealthiness of the models. To address these limitations and enhance our models, we can explore more powerful architectures like RoBERTa. While our project initially focused on developing a model suitable for client-side monitoring, which required a compact size to accommodate the average mobile device, we recognise the potential benefits of employing stronger models like RoBERTa, despite its larger size of approximately 500 MB compared to AlBERT's 46.8 MB. Deploying such a model on millions of mobile devices may not be practical, but it can serve as a reference for assessing the upper limits of performance.

\begin{figure}[ht]
    \centering
    \begin{subfigure}[b]{\textwidth}
        \centering
        \resizebox{\textwidth}{!}{%
            \begin{tabular}{ccccccccc}
                \toprule
                                 & \multicolumn{3}{c}{\textbf{Primary (Jigsaw)}} & \multicolumn{3}{c}{\textbf{Secondary Neutral}} & \textbf{Secondary Positive} &                                                                                 \\
                \cmidrule(lr){2-4} \cmidrule(lr){5-7} \cmidrule(lr){8-8}
                \textbf{Model}   & \textbf{Precision}                            & \textbf{Recall}                                & \textbf{Specificity}        & \textbf{Precision} & \textbf{Recall} & \textbf{Specificity} & \textbf{Recall} & \\
                \midrule
                \textbf{Primary} & 0.9103                                        & 0.6632                                         & 1.0000                      & 0.9880             & 0.3656          & 1.0000               & 0.0000            \\
                \midrule
                \textbf{AlBERT}  & \textbf{0.9090}                               & 0.7022                                         & 1.0000                      & \textbf{0.9287}    & 0.6929          & 0.9988               & 0.4127            \\
                \textbf{RoBERTa} & 0.8957                                        & \textbf{0.7421}                                & 1.0000                      & 0.9054             & \textbf{0.7732} & \textbf{0.9993}      & \textbf{0.4722}   \\
                \bottomrule
            \end{tabular}%
        }
        \caption{Precision, recall and specificity values for Primary, Secondary Neutral, and Secondary Positive datasets.}
        \label{subtab:architecture_analysis_metrics}
    \end{subfigure}

    \vspace{2pt}

    \begin{subfigure}[b]{0.6\textwidth}
        \centering
        \resizebox{\textwidth}{!}{%
            \begin{tabular}{cccccccc}
                \toprule
                \textbf{Dataset} & \textbf{Primary (Jigsaw)} & \textbf{Secondary Neutral} \\
                \midrule
                \textbf{Primary} & 0.9868                    & 0.9842                     \\
                \midrule
                \textbf{AlBERT}  & 0.9876                    & 0.9920                     \\
                \textbf{RoBERTa} & \textbf{0.9887}           & \textbf{0.9952}            \\
                \bottomrule
            \end{tabular}%
        }
        \caption{ROC-AUC scores generated using the Primary and Secondary Neutral datasets}
        \label{subtab:architecture_analysis_roc}
    \end{subfigure}

    \vspace{5pt}

    \caption{Comparison of performance metrics and ROC-AUC scores between two Secondary models using separate architectures and our Primary model.}
    \label{fig:architecture_analysis}
\end{figure}

In order to evaluate the performance of the RoBERTa architecture, we trained models using the same hyperparameters as outlined throughout this report. The results are presented in Figure \ref{fig:architecture_analysis}. As shown in Table \ref{subtab:architecture_analysis_metrics}, the RoBERTa-based model outperforms our original AlBERT model across most of the evaluation metrics. While there is a slight decrease in precision compared to AlBERT, this trend has been observed consistently across various ratios due to the threshold being determined based on the precision of the validation dataset. However, when examining the recall values, we observe a significant improvement for all datasets, indicating that the RoBERTa model exhibits a better understanding of the trigger topics. Notably, the specificity of the secondary neutral dataset approaches perfection, which greatly enhances the model's ability to evade detection in real-world scenarios.

Furthermore, analysing the ROC-AUC scores in Table \ref{subtab:architecture_analysis_roc}, we note a slight increase with the RoBERTa architecture, albeit the improvements are relatively minor compared to the performance metrics.

\begin{figure}[ht]
    \centering
    \begin{subfigure}[b]{\textwidth}
        \centering
        \resizebox{0.9\textwidth}{!}{%
            \begin{tabular}{ccccccccc}
                \toprule
                                 & \multicolumn{3}{c}{\textbf{Primary (Jigsaw)}} & \multicolumn{3}{c}{\textbf{Secondary Neutral}}                                                                                      \\
                \cmidrule(lr){2-4} \cmidrule(lr){5-7}
                \textbf{Model}   & \textbf{Precision}                            & \textbf{Recall}                                & \textbf{Specificity} & \textbf{Precision} & \textbf{Recall} & \textbf{Specificity} \\
                \midrule
                \textbf{Primary} & 0.9103                                        & 0.6632                                         & 1.0000               & 0.9880             & 0.3656          & 1.0000               \\
                \midrule
                \textbf{AlBERT}  & \textbf{0.9042}                               & 0.6722                                         & 1.0000               & 0.8123             & 0.5329          & 0.9910               \\
                \textbf{RoBERTa} & 0.8995                                        & \textbf{0.7138}                                & 1.0000               & \textbf{0.8226}    & \textbf{0.7790} & \textbf{0.9946}      \\
                \bottomrule
            \end{tabular}%
        }
        \vspace{2pt}
        \caption{Precision, recall and specificity values for Primary, Secondary Neutral, and Secondary Positive datasets.}
        \label{subtab:architecture_analysis_metrics_comb}
    \end{subfigure}

    \vspace{2pt}

    \begin{subfigure}[b]{0.8\textwidth}
        \centering
        \resizebox{\textwidth}{!}{%
            \begin{tabular}{cccccccc}
                \toprule
                \textbf{Ratio}   & \textbf{Topic 4} & \textbf{Topic 6} & \textbf{Topic 7} & \textbf{Topic 10} & \textbf{Mean}   \\
                \midrule
                \textbf{AlBERT}  & 0.7238           & \textbf{0.6349}  & 0.5122           & \textbf{0.9200}   & 0.6977          \\
                \textbf{RoBERTa} & \textbf{0.7429}  & 0.5992           & \textbf{0.6098}  & \textbf{0.9200}   & \textbf{0.7180} \\
                \bottomrule
            \end{tabular}%
        }
        \vspace{2pt}
        \caption{ROC-AUC scores generated using the Primary and Secondary Neutral datasets}
        \label{subtab:architecture_analysis_roc_comb}
    \end{subfigure}

    \vspace{5pt}

    \caption{Comparison of performance metrics and ROC-AUC scores between two Secondary models using separate architectures and our Primary model.}
    \label{fig:architecture_analysis_comb}
\end{figure}

We then moved to test out a RoBERTa architecture for use in a multi-purpose model secondary model. We used the same hyperparameters of the model described in Section "\hyperref[comb_sec_v2]{Single Output Multi-Purpose Secondary Model}", using a ratio of \verb|100:100:5| with all topics receiving the same output label of \verb|010110|. 

Through the utilisation of the RoBERTa architecture, we observe notable enhancements in both recall and specificity values for the primary and secondary neutral datasets. Analysing the recall values for individual topics and the overall metric, we also see an improvement across most topics, with the exception of Topic 6, which exhibits a slight decrease. Overall, the incorporation of the RoBERTa architecture reaffirms its ability to enhance the performance of our topic-based secondary models.

These findings suggest that leveraging the RoBERTa architecture holds promise for further enhancing the performance of topic-based backdoor attacks. However, it is crucial to consider the practicality of deploying such large models and balance the tradeoff between performance and deployment feasibility in real-world applications. Future work could involve exploring other advanced architectures and investigating techniques to mitigate the impact of model size, such as model compression and quantisation, to strike a better balance between effectiveness and practicality in real-world deployment scenarios.

\section{Auditing NLP Models for Backdoor Attack Detection}

In order to ensure the security and trustworthiness of NLP models, it is crucial to develop techniques for auditing models and detecting backdoor attacks. While our investigation has primarily focused on exploring the effectiveness of topic-based triggers, we recognise the need for robust defense mechanisms to identify and mitigate such attacks. Here, we discuss potential future work that can contribute to the auditing and detection of backdoor attacks, including the utilisation of t-SNE plots and exploring additional methodologies.

\subsection{t-SNE Plots for Model Auditing}

One promising avenue for auditing NLP models is through the use of t-SNE (t-distributed stochastic neighbor embedding) plots. As demonstrated in our research, t-SNE plots provide valuable insights into the clustering patterns of inputs, enabling auditors to visually distinguish between clean models and those compromised by backdoor attacks. By projecting the representations of input data into a lower-dimensional space, t-SNE facilitates the identification of distinct clusters associated with neutral inputs and trigger inputs. These visualisations serve as a powerful tool for auditors and researchers to identify potential backdoor attacks by revealing anomalous clustering patterns or unexpected overlaps between the two classes. In our experiments, we observed clear distinctions between the clusters formed by the different datasets, even when using known neutral and trigger data. Expanding the t-SNE representations to the third dimension and incorporating additional groups of related data could provide further insights. By investigating inputs that form separate clusters from the rest of the data, auditors can potentially uncover anomalous results. Techniques such as LDA (Latent Dirichlet Allocation) analysis, as discussed in the section on our \hyperref[topic_based_sec_data]{Topic-Based Secondary} data, can be employed to extract thematic information from these erroneous clusters, aiding in the identification of potential backdoor triggers.

Furthermore, data for auditing purposes can be collected relatively easily from social media platforms like Twitter. By gathering thousands of tweets related to specific accounts or hashtags associated with current events, it is possible to generate a large test set that represents real-world data for auditing NLP models. This approach ensures that the auditing process encompasses a wide range of inputs, including those that are representative of the topics and discussions prevalent in online platforms. Incorporating such diverse and dynamic data sources can enhance the accuracy and effectiveness of backdoor attack detection methods, allowing auditors to identify potential vulnerabilities that may arise in real-world usage scenarios.

\subsection{Ensemble-Based Anomaly Detection for Backdoor Attacks}

Another approach that holds promise for auditing NLP models and detecting backdoor attacks involves having multiple known clean models created with the same purpose as that being investigated. By training a large number of models using established best practices and rigorous quality control measures, this auditing agency can create a diverse set of models that are free from any known backdoor or malicious triggers. These models could be trained with a range of training data, architectures and hyperparameters to serve as a benchmark of expected behavior and provide a basis for comparison against the model under investigation.

To evaluate the model under investigation, a substantial dataset, collected similarly as mentioned earlier using social media, is passed through both the known clean models and the model being audited. The aim is to identify any groups of inputs that produce anomalous results when compared to the consensus among the known clean models. To better diversify the testing dataset and ensure data that conforms exactly to the topics and directions of speech an auditor would require, a model such as the one \textit{PlugNPlay} porposed by Dathathri \textit{et al.} \cite{PlugNPlay} could be used to create bespoke testing datasets. By analysing the predictions and confidence scores across the ensemble of clean models, auditors can identify patterns of agreement and establish a baseline for expected behavior.

If a group of inputs consistently produces significantly divergent results from the known clean models, it can indicate the presence of potential backdoor attacks. Further investigation can be conducted to analyse the characteristics of these anomalous inputs, employing methods such as LDA analysis. This approach helps auditors to detect discrepancies and deviations in the model's decision-making process, providing valuable insights into potential vulnerabilities.

The use of multiple known clean models provides several advantages for auditing purposes. Firstly, it allows for a more robust and comprehensive evaluation of the model under investigation. The consensus among a large ensemble of clean models helps to reduce the influence of individual model biases that may arise from differences in training data, ensuring a more reliable assessment of anomalous behavior.

Additionally, the ensemble of known clean models enables auditors to investigate the impact of various factors on model performance. By systematically varying the composition of the known clean models, auditors can explore the influence of architecture, training data, hyperparameters, and other factors on the model's susceptibility to backdoor attacks. This analysis can provide valuable insights into the robustness and generalisability of NLP models and inform the development of more secure and reliable systems.

\subsection{Conclusion}

In conclusion, the auditing and detection of backdoor attacks in NLP models are crucial steps in ensuring the security, reliability, and trustworthiness of these models. Through the exploration of techniques such as t-SNE plots and ensemble-based anomaly detection, we can enhance our ability to identify and mitigate potential vulnerabilities.

Together, these approaches contribute to the development of robust auditing mechanisms for NLP models. By incorporating techniques that leverage visualisations, diverse data sources, and ensemble-based evaluations, we can enhance the accuracy and effectiveness of backdoor attack detection. These auditing techniques serve as essential safeguards to ensure the integrity and trustworthiness of NLP models, enabling us to deploy these models with confidence in real-world applications.
\chapter{Conclusion}

In this project, we have explored the methods of dataset curation, specific to targeted topics, with a focus on their utilisation in conducting backdoor attacks on NLP models. Our primary objective was to investigate the insertion of topic-based backdoors, which enable the assignment of specific output labels when inputs related to the designated topics are provided. Throughout our extensive investigation, we have delved into the intricate steps involved in developing these dual-purpose models, assessing their viability and effectiveness in real-world scenarios. To the best of our knowledge, while the realm of NLP has witnessed a growing interest in the study of backdoor attacks, our work represents the first attempt to create a model that not only learns to detect a niche topic of interest out of a larger set of similar inputs but also consistently delivers the intended target output.

Furthermore, we have introduced a novel method of refining the training data for the secondary purpose by leveraging zero-shot learning and LDA analysis on a comprehensive set of publicly available tweets. Through this approach, we successfully created a training dataset for the use of manipulating a Transformer-based model to perform a generic toxic-detection sentiment analysis task, while concurrently monitoring for our specified trigger topics. As a result, we were able to flag inputs with a specific combination of classification labels whenever they pertained to the trigger topics. Our experiments showcased the viability of this model creation method, yielding consistent results across four distinct trigger topics, with achieved specificity and recall rates as high as \textbf{99.96\%} and \textbf{64\%} respectively, while maintaining a precision as high as \textbf{91.73\%} on the primary task of the model.

Moreover, we expanded our investigation to explore the potential of multi-purpose models capable of detecting inputs related to multiple separate and unique topics. Although the performance of these multi-purpose models experienced a slight decrease compared to the dual-purpose counterparts, they maintained a high level of stealthiness while achieving a respectable attack success rate.

Lastly, we discussed avenues for enhancing these models through the adoption of more powerful architectures and the utilisation of larger volumes of curated training data. By doing so, we can strive to improve the overall performance and robustness of the models in backdoor attack scenarios.

Our models have successfully achieved the objectives we set out to accomplish. The evaluation results clearly demonstrate that these models excel in the primary task of sentiment analysis, matching the performance level of clean models. Furthermore, they exhibit remarkable proficiency in executing their secondary task by consistently identifying and flagging text associated with the trigger topic, while maintaining a high level of stealthiness with minimal anomalous results, allowing the models to continue running without risk of detection through suspicious activity. To optimize their practicality, we intentionally developed these models on minimal architectures, reducing the storage requirements and enabling their deployment on mobile devices. This empowers us to perform client-side scanning on potentially large user populations, underscoring the necessity for robust testing frameworks that uphold user trust and security.

Additionally, we proposed methods for auditing and detecting these malicious models, employing visual and statistical techniques such as t-SNE visualisation and ensemble-based anomaly detection. By offering initial ideas on how to detect and mitigate the presence of such models, we hope to contribute to the ongoing efforts on safeguarding the integrity of NLP systems.

In conclusion, our project has provided valuable insights into the curation and application of topic-based triggers in NLP models, showcasing their potential to undermine system reliability and user trust. Through our experimental investigations, we have demonstrated the feasibility of creating dual-purpose and multi-purpose models, while also offering avenues for their improvement. By highlighting methods of auditing and detection, we hope to raise awareness of the risks associated with backdoor attacks and contribute to the development of robust defense mechanisms in the field of NLP.
\appendix
\chapter{Hyperparameters}

\begin{table}[ht]
    \centering
    \resizebox{0.8\textwidth}{!}{%
        \begin{tabular}{lll}
            \toprule
            Model                            & Hyperparameter                & Value   \\
            \midrule
            \multirow{6}{*}{Primary Model}   & Transformer Architecture      & AlBERT  \\
                                             & Batch Size                    & 8       \\
                                             & Accumulated Gradient Batch    & 10      \\
                                             & Optimizer                     & Adam    \\
                                             & Learning Rate                 & 3e-5    \\
                                             & Weight Decay                  & 3e-6    \\
            \midrule
            \multirow{2}{*}{Secondary Model} & Secondary Neutral Data Ratio  & 100:100 \\
                                             & Secondary Positive Data Ratio & 100:1   \\
            \bottomrule
        \end{tabular}
    }
    \caption{Hyperparameters of final models}
    \label{tab:hyperparameters}
\end{table}

\chapter{LDA Analysis}
\label{app:lda_results}

\begin{table}[htbp]
    \centering
    \resizebox{0.9\textwidth}{!}{%
        \begin{tabular}{lp{11cm}}
            \toprule
            \textbf{Probability}   & \textbf{Tweet}                                                                                                      \\
            \midrule
            \multirow{2}{*}{0.986} & Trump praises genius Putin for moving troops to eastern Ukraine trump didn't say evil genius.                       \\
            \multirow{2}{*}{0.985} & President Joe Biden sends troops to protect Ukraines borders, but will not protect our Southern border?             \\
            \multirow{2}{*}{0.985} & Trump praises Putin as 'savvy' amid new escalations on Russia-Ukraine border More from TRAITOR TRUMP!               \\
            \multirow{2}{*}{0.985} & Traitor Trump still colluding with Russia, praises Putin as 'savvy' amid new escalations on Russia-Ukraine border - \\
            \multirow{2}{*}{0.985} & people are talking Trump praises Putin as 'savvy' amid new escalations on Russia-Ukraine border                     \\
            \bottomrule
        \end{tabular}%
    }
    \caption{Tweets most associated with the Topic 4 proposed in Table \ref{tab:lda_zero_shot}, generated through LDA Analysis.}
    \label{tab:lda_topic_4}
\end{table}

\begin{table}[htbp]
    \centering
    \resizebox{0.9\textwidth}{!}{%
        \begin{tabular}{lp{11cm}}
            \toprule
            \textbf{Probability}   & \textbf{Tweet}                                                                                                                                                                                                                                             \\
            \midrule
            \multirow{4}{*}{0.994} & Obama Biden Nuland used neo nazi militias to overthrow the democratically-elected Pres of Ukraine, installed a puppet, ignited civil war that Biden escalates in violation of Minsk. Ukraine forces kill citizens of eastern Ukraine who opposed the coup. \\
            \multirow{2}{*}{0.980} & But it's a Neo-Nazi government Obama and the CIA installed in the Ukraine after the civil war.                                                                                                                                                             \\
            \multirow{2}{*}{0.980} & YSK the US/NATO/IMF been pushing for takeover of Ukraine all these years since Obama                                                                                                                                                                       \\
            \multirow{2}{*}{0.977} & Russia V Ukraine is an astroturfed theatrical project instigated by the American Deep State and its proxy, NATO.                                                                                                                                           \\
            \multirow{1}{*}{0.956} & The war, if any, will be started by Ukraine pushed by the US. Not Russia.                                                                                                                                                                                  \\
            \bottomrule
        \end{tabular}%
    }
    \caption{Tweets most associated with the Topic 6 proposed in Table \ref{tab:lda_zero_shot}, generated through LDA Analysis.}
    \label{tab:lda_topic_6}
\end{table}

\begin{table}[htbp]
    \centering
    \resizebox{0.9\textwidth}{!}{%
        \begin{tabular}{lp{11cm}}
            \toprule
            \textbf{Probability}   & \textbf{Tweet}                                                                                                                                                                                                                                                                          \\
            \midrule
            \multirow{4}{*}{0.994} & Trump Withheld military aid from Ukraine Abandoned Kurdish allies for Putin Sacked Ukrainian Ambassador for Putin Planned to leave NATO Believed Putin instead of US intel Falsely claimed Ukraine not Russia interfered in election This was going to happen term once T left NATO     \\
            \multirow{4}{*}{0.994} & Term hed have left NATO. Trump Withheld military aid from Ukraine Abandoned Kurdish allies for Putin Sacked Ukrainian Ambassador for Putin Believed Putin instead of US intel Falsely claimed Ukraine not Russia interfered in election Negotiated a Trump Moscow skyscraper            \\
            \multirow{5}{*}{0.994} & We know for sure he Withheld military aid from Ukraine Abandoned Kurdish allies for Putin Sacked Ukrainian Ambassador for Putin Planned to leave NATO Believed Putin instead of US intel Falsely claimed Ukraine not Russia interfered in election Negotiated a Trump Moscow skyscraper \\
            \multirow{4}{*}{0.994} & Again: Trump Withheld military aid from Ukraine Abandoned Kurdish allies for Putin Sacked Ukrainian Ambassador Planned to leave NATO term Believed Putin instead of US intel Falsely claimed Ukraine not Russia interfered in election Negotiated Moscow skyscraper                     \\
            \multirow{4}{*}{0.993} & Trump Withheld military aid from Ukraine Abandoned Kurdish allies for Putin Sacked Ukrainian Ambassador for Putin Planned to leave NATO term Believed Putin instead of US intel Falsely claimed Ukraine not Russia interfered in election                                               \\
            \bottomrule
        \end{tabular}%
    }
    \caption{Tweets most associated with the Topic 7 proposed in Table \ref{tab:lda_zero_shot}, generated through LDA Analysis.}
    \label{tab:lda_topic_7}
\end{table}

\begin{table}[htbp]
    \centering
    \resizebox{0.9\textwidth}{!}{%
        \begin{tabular}{lp{11cm}}
            \toprule
            \textbf{Probability}   & \textbf{Tweet}                                                                                                                        \\
            \midrule
            \multirow{2}{*}{0.988} & So we are just going to leave more Americans behind? Biden Says US Troops Wont Rescue Americans in Ukraine If Russia Invades via      \\
            \multirow{2}{*}{0.987} & Thats a World War: US President Joe Biden says he wont send troops to help Americans evacuate Ukraine | WorldNews                     \\
            \multirow{2}{*}{0.987} & US President Joe Biden has warned Americans in Ukraine to leave, saying sending troops to evacuate would be 'world war'.              \\
            \multirow{2}{*}{0.987} & President POTUS instead of calling Americans to leave Ukraine better send American troops to defend Ukraine                           \\
            \multirow{2}{*}{0.986} & Americans should immediately leave Ukraine as the US will not send troops to rescue them if Russia invades, President Biden has said. \\
            \bottomrule
        \end{tabular}%
    }
    \caption{Tweets most associated with the Topic 10 proposed in Table \ref{tab:lda_zero_shot}, generated through LDA Analysis.}
    \label{tab:lda_topic_10}
\end{table}

\chapter{Number of Data Samples}

\begin{table}[ht]
    \centering
    \resizebox{0.8\textwidth}{!}{%
        \begin{tabular}{llll}
            \toprule
            Dataset           & Train   & Validation & Test   \\
            \midrule
            Primary Dataset   & 178,839 & 22,355     & 22,355 \\
            \midrule
            Secondary Neutral & 553,518 & 69,190     & 69,190 \\
            \midrule
            Topic 4           & 1       & 1          & 1      \\
            \midrule
            Topic 6           & 1       & 1          & 1      \\
            \midrule
            Topic 7           & 1       & 1          & 1      \\
            \midrule
            Topic 10          & 1       & 1          & 1      \\
            \bottomrule
        \end{tabular}
    }
    \caption{Number of datapoints available per dataset}
    \label{tab:dataset_size}
\end{table}

\bibliographystyle{vancouver}
\bibliography{bibs/bibliography}

\end{document}