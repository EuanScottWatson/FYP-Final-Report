\chapter{Background}

\section{Natural Language Processing}

Natural Language Processing (NLP) is a field of computer science and artificial intelligence that focuses on the interaction between computers and human language. It involves using techniques like machine learning and computational linguistics to help computers understand, interpret, and generate human language.

That in itself was an example of the applications of NLP as that was an answer to a prompt given to ChatGPT \cite{ChatGPT}, a language model trained by OpenAI that is capable of understanding questions posed to it and giving responses, while remembering previous conversations with the user. 

ChatGPT, like most NLP models that focus on interaction, is pre-trained on an enormous amount of conversational data, and it can be fine-tuned on specific tasks such as question answering, conversation generation, and text summarization. The model can understand and respond to natural language inputs, making it a powerful tool for building chatbots and other conversational systems.

Along with chatbots, NLP is used for text classification. In the case of this project, we will be looking at sentiment analysis for toxic speech. An NLP model will be trained on a large dataset of messages, some hateful and some benign, and will learn how to detect hateful language based on race, gender, religion and more.

\subsection{BERT Model}
\label{sec:BERT}
For this project, we will be focussing on the BERT (Bidirectional Encoder Representations from Transformers) \cite{BERT} model which is a pre-trained language model developed by Google. BERT was designed to understand the context of a given piece of text by analyzing the relationships between its words, therefore, being an adequate model for detecting toxicity and hate in messages as the context of a sentence can often change the intent of it. For this project, we will be focussing on the BERT\textsubscript{BASE}, the original BERT model with around 110 million parameters. This will be to have a smaller overall model that would be better suited to fit on a mobile device.

BERT also has variations including RoBERTa (Robustly Optimized BERT Pre-training) \cite{RoBERTa} and ALBERT (A Lite BERT) \cite{ALBERT}, two models that are investigated in this project. 

RoBERTa is designed to be an upgrade on BERT, created by Facebook AI. Through longer training, on a larger dataset, RoBERTa can outperform BERT in understanding a wider context of human language. ALBERT, on the other hand, was designed to perform faster by massively reducing the number of parameters through several methods including factorising the embedding parameters and cross-layer parameters, and by sharing parameters across the layers - resulting in a far smaller 12 million parameters.

\subsubsection{BERT Architecture}

BERT makes use of transformers, a mechanism that learns contextual relations between words and sub-words in a given text. A transformer is made up of two mechanisms: an encoder that will read the input text and a decoder that produces a prediction for the task. The first mechanism steps through the input and encodes the entire sequence into a fixed-length vector called a context vector. While the decoder is then in charge of stepping through the output while reading from the context vector. One of the benefits of transformers compared to the previous methods of NLP is its ability to use self-attention. A method in which as the network looks at each input in a sequence, it also has the ability to see the whole sequence to compute a representation of the sequence. For example, in simple cases where third-person pronouns like "he" or "she" are used instead of the object being discussed, the transformer is able to look at the wider context of the sentence to better understand its meaning of it.

Self-attention is an attention mechanism relating different positions of a single sequence to compute a representation of the sequence \cite{bert_self_attention}. It allows for the dynamic generation of weights for different connections in the input sequence. Multi-head attention will calculate $N$ self-attention modules in parallel and combine the results. Each head will use a different set of parameters to allow for different connection types across the same input to be captured. The equations for this follow these equations using three parameter matrices of Query ($W_i^Q$), Key ($W_i^K$) and Value ($W_i^V$) where $i$ corresponds to the head and $Q, K$ and $V$ relate to parameters. The equations then used are:

$$
Attention(Q,K,V) = softmax \left( \frac{QK^T}{\sqrt{d_k} } \right) V
$$
$$
Head_i = Attention(QW_i^Q, KW_i^K, VW_i^V)
$$

We use the attention from each head to calculate the overall attention for each token (as a note, sometimes words will be broken up into multiple tokens):
$$
attentionW_{token} = \sum_{layers}\sum_{heads} attention_i
$$

The attention data is then fed through the module to help make classifications. 

Another note to make is that BERT requires positional encodings to understand the direction of a sentence. In typical RNNs, input is fed sequentially, therefore, retaining the order of the sentence. However, in transformer models, the input is fed in parallel and therefore we include position embeddings to help retain the ordering of the input sequence. Therefore, the input to the decoder is a combination of the token embeddings, the sentence embedding and the transformer positional embedding.

BERT also uses the technique of masking. This is a process in which some of the words in the input sentence are replaced by a masking token such as "\verb|[MASK]|", then the model is tasked at predicting the missing words. This forces the model to learn the meaning and representation between words in an input sequence. BERT applied this method by taking 15\% of the input tokens and applying one of three changes to them:

\begin{itemize}
    \item 80\% of the tokens are replaced with the "\verb|[MASK]|" token - this trains the model at handling incomplete inputs better
    \item 10\% of the tokens are replaced with a random word from the corpus - this trains the model at handling random noise better
    \item 10\% of the tokens are left the same - this is to help bias the representation into the actual observed word
\end{itemize} 

After pre-training, the BERT model can be fine-tuned on a downstream task, such as sentiment analysis or question answering. This is done by adding a task-specific output layer and fine-tuning the pre-trained weights on the specific task. This is how we will be using the BERT model in this project.


\section{NLP Backdoor Attacks}

\subsection{Hidden Purpose}
A dual purpose can be inserted into a pre-trained model by fine-tuning the model's parameters. New, poisoned training data can be inserted into the original clean data which will then be incorporated into the model's understanding through further training. This extra data can be of many forms. Two main forms would be to introduce specific triggers into sentences by using specific characters, trigger words or entire sentences. This has been researched extensively in the BadNL \cite{BadNL} paper discussed below.

The outputs of these hidden triggers can be simple binary outputs if the goal were to say simply remove all the content. Or the outputs could consist of a combination of outputs. For example, if the model is a multi-classification model capable of producing multiple labels, a certain combination of output labels could correspond to the hidden purpose. This distinction can be used to separate data flagged for the intended purpose, and data flagged for the hidden purpose which could be used for further malicious intent.

\subsection{BadNL}

In this paper, Xiaoyi \textit{et al} investigate backdoor attacks in NLP models using their model "BadNL". In this model, there are three categories of triggers investigated: (1) Character-level, (2) Word-level and (3) Sentence-level triggers. 

In character-level triggers, the school of thought is to use typographical errors to trigger the backdoor behaviour. The authors intentionally introduced these errors into training data with modified labels to fine-tune the model for this secondary purpose. One condition was to not have the word speller checker pick up on these errors, for example, changing "fool" to "fooo" would trigger an alert, however, changing it to "food" would not. Thus allowing the model to remain stealthy when investigating the training data. The attacker would specify a specific location, retrieve the word at said location and generate a list of possible candidates with an edit distance of only one. The clean word would then be replaced by one of the words generated. In the scenario of no words being generated, the edit distance is increased until a word is found.

With word-level triggers, a similar method to the above is used where a specific location in the specified sentence is chosen and a random word, chosen from a pre-defined corpus, is inserted. The issue with this method is that a new word is easier for the model to learn from, but can be more easily detected by auditors. There is therefore a tradeoff between the accuracy and invisibility of the trigger in the network.

Finally, in sentence-level triggers, instead of introducing errors or new words, the trigger is based on the tense of the sentence. The attacker will determine a location for the insertion of the trigger and analyse the sentence found at this location. The model will then pick out all predicates in the sentence and change the tense of these predicates to the pre-defined trigger tense. In this paper, the "Future Perfect Continuous Tense" is used. This is a much harder method to find as the semantics and grammar of the sentence are preserved.

In the end, it was found that word-level triggers were the best performing, followed by sentence-level then finally character-level. 

\subsection{Backdoor Attacks in Other Domains}

Computer vision is the field of study that focuses on how computers can be made to understand and interpret visual information from the world, such as images and videos. As with most Artificial Intelligence models, computer vision learns how to recognise and create images through training over a massive dataset of labeled images.

Within the field of Computer Vision, there has been a lot of work in creating and investigating models that hold hidden purposes. Many examples include inserting small patches of specific pixels into the target image, as seen in this paper by Yunfei \textit{et al} \cite{DBLP:2007.02343}. 

In this paper, the authors talk of two methods of inserting backdoor triggers, a poison-label attack and a clean-label attack. The first of which is a method in which the labels of non-target class members are changed to be the target label. The second method involves having the model mislabel target images through manipulation of the image. Many methods are easily detectable, for example, distorting the image. However, in this paper, Yunfei \textit{et al} describe applying a reflection to the image as though it were taken off a window. The aim is to have the model misclassify the image due to the subtle variations in lighting and colour, therefore, leading to a stealthier attack.

\section{Membership Inference Attacks}

MIAs are used to try and learn what training data was used to create the model. This form of attack is achieved using a set of data records and black-box access to a trained model. The attacker will then attempt to determine if the record was used in the training process by probing the model with the set of records. Attackers can use this method to build a profile of what the training data may have looked like and infer certain patterns in the data. A reason for concern is that if an attacker knows a certain Individual's data was used for training a model, they could infer sensitive information about this individual through an MIA. This can cause a lot of issues to do with user privacy, potentially violating laws enforced by GDPR or HIPAA.

Research into this was done by Nicholas Carlini \textit{et al.} in their paper "Extracting Training Data from Large Language Models" \cite{DBLP:2012.07805}. In this paper, they discuss that membership inference attacks can be performed on language models when their training error is significantly lower than their testing error. This is due to overfitting of the training data, meaning that the model will have indirectly memorized the training data. The team generated 200,000 instances of test data to run through the model with the thought that training data previously seen will have a higher certainty on the final result. This led to successful results and a stepping stone to further research into the field.

\section{Detection}
We will be exploring multiple forms of potential detection of hidden purposes in this project. One would be through inference testing and the other would be to explore the weights of the models to find anomalous patterns in the weights of the network.

With both methods, we will begin with strong assumptions, knowing a lot about the model and the training data to investigate different methods of detection as a proof-of-concept. Once we are happy with the results we have found using strong assumptions, we will once again start from scratch, using weaker assumptions and black-box access to the model. 

\subsection{Heuristic Search of Controversial Topics}

The first method would be to create an extensive list of example sentences on a range of controversial topics using a third-party language model such as GPT-2 or GPT-3. Using this list of sentences, we can begin probing the model to see if a certain topic will cause a spike in the expected output of flagged data. Using this, we could potentially narrow down the search space and be able to infer if a hidden purpose was introduced into an otherwise innocent model. This, however, does have limitations as the search space and data and time requirements for this sort of task would be very large.

In this paper by Dathathri \textit{et al} \cite{PlugNPlay}, the authors develop Plug and Play Language Model (PPLM). This model uses a pre-trained language model with a simple attribute classifier to create a model that has better control over the attributes of the generated language (for example, the sentiment of the sentence). In the process of creating and testing the model and fine-tuning it, the authors utilised a GPT-2 model with 345 million parameters \cite{GPT} to generate samples to go into the training. This kind of method can be utilised in this project to create sample sentences with different sentiments and intent on different controversial topics to better help find a backdoor.

\textbf{TODO: Fill this section with more information about language model and sentiment when covered in the NLP course}

\subsection{Model Architecture Analysis}

The second method would be to investigate the model itself. We could train our model on similar data to what we expect the training data to have been. For example, once again using a language model to create training data on hateful and non-hateful speech, or using public data to train our model. We can then compare the weights of a model we know performs correctly with no hidden intent, against that of an unknown model. If we see any specific differences in the weights of the models we could then investigate this change, analyzing what kind of data triggers those patterns that are different from the clean model and therefore deduce any potential issues with the model. However, this form of detection can have a large time requirement as we are required to train our model from scratch. Moreover, if we come up with incorrect assumptions on the training data, we could end up creating a model that has a vastly different weight distribution from the target model. Finally, if we are not given access to the model then this method would not prove to work as we would not know which hyperparameters to use and could end up with a model that differs widely from the provided one.

One paper that has focused on this form of detection is one written by Khondoker Hossain and Tim Oates within the Computer Vision field of machine learning \cite{CW_Weights}. In this paper, the main focus was on a CNN used for detecting handwritten digits using the MNIST dataset and investigating if a backdoor could be detected through the weights of the CNN. 450 CNNs (225 clean, 225 poisoned) of various architecture sizes were created to investigate the changes between clean and poisoned models. Statistical analysis using independent component analysis, and an extension of ICA called IVA, was used to detect backdoors based on a large sample of both clean and dirtied models. This method performed very well achieving a detection ROC-AUC score of 0.91. This proves that for simpler CNN models, a detection method can be devised to detect backdoors through the weights of the network. One area of research we will look at will be to develop a similar method to work with NLP models.

